
\subsection{Morphism of schemes}
\Definition{
  \index{すきーむ@スキーム!のしゃ@の射}\index{scheme!morphism}
  $f:X\to Y$が\textbf{スキームの射(morphism of schemes)}とは局所環付き空間としての射とする.
}

環の射$\varphi:A\to B$が誘導する射$\spec{B} \to \spec{A}$を$\varphi^{a}$と書くことにする.\\
つまり$\mathfrak{p}\in \spec{B}$に対して$\varphi^{a}(\mathfrak{p}) = \varphi^{-1}(\mathfrak{p})$

\Proposition{
  $\varphi: A \to B$を環の射とする.このとき
  \begin{equation*}
    (\varphi^{a},(\varphi^{a})^{\#}):\spec{B} \to \spec{A}
  \end{equation*}
  は$(\varphi^{a})^{\#}(\spec{A})=\varphi$を満たすスキームの射である.
}{
  $X=\spec{B},Y=\spec{A}$と置く.任意の$f\in A$に対して
  \begin{equation*}
    (\varphi^{a})^{-1}(D(f))=D(\varphi(f))
  \end{equation*}
  が成り立ち,実際
  \begin{align*}
    (\varphi^{a})^{-1}(D(f)) 
    &= \{ \mathfrak{p} \in X\ |\ \varphi^{a}(\mathfrak{p}) \in D(f)\}\\
    &= \{ \mathfrak{p} \in X\ |\ f \notin \varphi^{a}(\mathfrak{p})\}\\
    &= \{ \mathfrak{p} \in X \mid f \notin \varphi^{-1}(\mathfrak{p})\}\\
    &= \{ \mathfrak{p} \in X \mid \varphi(f) \notin \mathfrak{p}\}\\
    &= D(\varphi(f))
  \end{align*}  
  である.$\varphi$から誘導される環の射
  \begin{equation*}
    (\varphi^{a})^{\#}(D(f)):\mathcal{O}_{Y}(D(f)) = A_{f} \to B_{\varphi(f)} = \mathcal{O}_{X}(D(\varphi(f))) = (\varphi^{a})_{*}\mathcal{O}_{X}(D(f))
  \end{equation*}
  これは制限写像と可換(compatibleという意味で)になる.よって層の射
  \begin{equation*}
    (\varphi^{a})^{\#}:\mathcal{O}_{Y} \to \varphi^{a}_{*}\mathcal{O}_{X}
  \end{equation*}
  に拡張できる.
  %開基?(or 基本開集合族)で決まっていれば拡張できることを証明する.
  更に,任意の$\mathfrak{q} \in X$に対して$\varphi$から誘導される環の射
  \begin{equation*}
    (\varphi^{a})^{\#}_{\mathfrak{q}}:A_{\varphi^{a}(\mathfrak{q})} \to B_{\mathfrak{q}}
  \end{equation*}
  は局所射で,実際
  \begin{align*}
    (\varphi^{a})^{\#}_{\mathfrak{q}}(\varphi^{a}(\mathfrak{q})A_{\varphi^{a}(\mathfrak{q})})
    &= \{ \varphi(a)/\varphi(p) \mid a\in \varphi^{a}(\mathfrak{q}),p \notin \varphi^{a}(\mathfrak{q})\}\\
    &= \{\varphi(a)/\varphi(p)\mid \varphi(a)\in \mathfrak{q},\varphi(p)\notin \mathfrak{q}\}\\
    &\subset \{b/q\mid b\in \mathfrak{q},q\notin \mathfrak{q}\}\\
    &= \mathfrak{q}B_{\mathfrak{q}}
  \end{align*}
  よって$(\varphi^{a},(\varphi^{a})^{\#})$は
  局所環付き空間の射になる.構成により
  \begin{equation*}
    (\varphi^{a})^{\#}(Y):\mathcal{O}_{Y}(Y) = A \to B = \mathcal{O}_{X}(X) = (\varphi^{a})_{*}\mathcal{O}_{X}(Y)
  \end{equation*}
  で$(\varphi^{a})^{\#}(Y) = \varphi$を満たす.
}

\Lemma{
  $A$を環とし,$I$をそのイデアルとする.このとき,スキームの射
  \begin{equation*}
    i:\spec{A/I} \to \spec{A}
  \end{equation*}
  が自然な射影$\varphi:A\to A/I$によって誘導される.$i$は$\im{i} = V(I)$へのスキームの閉はめ込みである.
  更に,任意の$\spec{A}$の基本開集合$D(f)$に対して
  \begin{equation*}
    (\ker{i^{\#}})(D(f)) = I\otimes_{A}A_{f}
  \end{equation*}
  が成り立つ.
}{
  $i$が閉はめ込みであることはよい.次に,任意の$\spec{A}$の基本開集合$D(f)$に対して先ほど
  みたように,標準的な全射
  \begin{equation*}
    \mathcal{O}_{\spec{A}}(D(f)) = A_{f} \to (A/I)_{\varphi(f)} = i_{*}\mathcal{O}_{\spec{A/I}}(D(f))
  \end{equation*}
  がある.これにより,$i^{\#}$の全射性と,
  \begin{align*}
    (\ker{i^{\#}})(D(f)) 
    &= \ker{(i^{\#}(D(f)))}\\
    &= \ker{(A_{f} \to (A/I)_{\varphi(f)})}\\
    &= I_{f} \\
    &= I\otimes_{A}A_{f}
  \end{align*}
  がわかる.
}

\Definition{
  $Z$を$X$の閉集合とする.このとき$Z$が
  \index{へいぶぶんすきーむ@閉部分スキーム}\index{closed subscheme}
  \textbf{閉部分スキーム(closed subscheme)}とは
  包含写像$j:Z\to X$が閉はめ込み
  \begin{equation*}
    (j,j^{\#}):(Z,\mathcal{O}_{Z}) \to (X,\mathcal{O}_{X})
  \end{equation*}
  となるときをいう.
}



\Proposition{
  $X=\spec{A}$をアフィンスキームとする.$j:Z\to X$をスキームの閉はめ込みとする.
  このとき,$Z$はアフィンスキームで,あるイデアル$J\subset A$が唯一存在して
  $j$は同型$Z\stackrel{\simeq}{\longrightarrow}\spec{A/J}$を誘導する.
}{}

\Definition{
  $S$をスキームとする.このとき$X$が
  \index{えすすきーむ/えすじょうのすきーむ@$S$-スキーム/$S$上のスキーム}
  \index{S-scheme/scheme over S@$S$-scheme/scheme over $S$}
  \textbf{$S$-スキーム($S$-scheme)}または\textbf{$S$上のスキーム(scheme over $S$)}とは
  スキームの射$\pi:X \to S$が与えられているときをいう.この$\pi$を
  \index{こうぞうしゃ@構造射}\index{structural morphism}
  \textbf{構造射(structural morphism,\ structure morphism)}
  ,$S$を
  \index{きていすきーむ@基底スキーム}\index{base scheme}
  \textbf{基底スキーム(base scheme)}という.$S=\spec{A}$のときまた$X$を
  \index{えーすきーむ/えーじょうのすきーむ@$A$-スキーム/$A$上のスキーム}\index{A-scheme/scheme over A@$A$-scheme/scheme over $A$}
  \textbf{$A$-スキーム($A$-scheme)}または\textbf{$A$上のスキーム(scheme over $A$)}
  という.このとき$A$を\index{きていかん@基底環}\index{base ring}\textbf{基底環(base ring)}
}

\Definition{
  $\pi:X\to S,\rho:Y \to S$を$S$上のスキームとする.このとき
  \index{えすすきーむ/えすじょうのすきーむ@$S$-スキーム/$S$上のスキーム!のしゃ@の射}
  \index{S-scheme/scheme over S@$S$-scheme/scheme over $S$!morphism}
  \textbf{$S$-スキームの射(morphism of $S$-scheme)}$f:X\to Y$とは$f$がスキームの射で
  $\rho \circ f = \pi$を満たすことをいう.
}

スキーム$X,Y$に対して
\begin{equation*}
  \hom{\textbf{Sch}}{X}{Y}:=\{f:X\to Y\ |\ f\text{ is morphism of schemes}\}
\end{equation*}
また,環$A,B$に対して
\begin{equation*}
  \hom{\textbf{Ring}}{A}{B}:=\{f:A\to B\ |\ f\text{ is morphism of rings}\}
\end{equation*}
とおく.このとき標準的な写像
\begin{equation*}
  \rho:\hom{\textbf{Sch}}{X}{Y} \to \hom{\textbf{Ring}}{\mathcal{O}_{Y}(Y)}{\mathcal{O}_{X}(X)}
\end{equation*}
がある.実際$(f,f^{\#})\in \hom{\textbf{Sch}}{X}{Y}$とすると
\begin{equation*}f^{\#}(Y):\mathcal{O}_{Y}(Y) \to f_{*}\mathcal{O}_{X}(Y) = \mathcal{O}_{X}(f^{-1}(Y))=\mathcal{O}_{X}(X)
\end{equation*}
がある.
%関手的であることを確認する.


\Definition{
  $\pi:X\to S$を$S$上のスキームとする.\textbf{$X$の切断(section of $X$)}\index{せつだん@切断}
  \index{section}とは$S$上のスキームの射$\sigma:S\to X$で$\pi \circ \sigma=\text{id}_S$となるときをいう.
  $X$の切断の集合を$X(S)$($S=\spec{A}$のときは$X(A)$)とかく.
}

\Example{
  $X$を体$k$上のスキームとする.このとき
  \begin{equation*}
    X(k) = \{x\in X\mid k(x) = k\}
  \end{equation*}
  実際$\sigma\in X(k)$をとる.$\mathcal{O}_{\spec{k},(0)} = k_{(0)} = k$より
  \begin{equation*}
    \sigma^{\#}_{(0)}:\mathcal{O}_{X,\sigma((0))}\to \mathcal{O}_{\spec{k},(0)} = k
  \end{equation*}
  で,
}{}
