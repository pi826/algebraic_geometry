
\subsection{Morphism of schemes}
\Definition{
  $f:X\to Y$が\textbf{スキームの射(morphism of schemes)}とは局所環付き空間としての射とする.
}

環の射$\varphi:A\to B$が誘導する射$\spec{B} \to \spec{A}$を$\varphi^{a}$と書くことにする.

\Proposition{
  $\varphi: A \to B$を環の射とする.このとき
  \begin{equation*}
    (\varphi^{a},(\varphi^{a})^{\#}):\spec{B} \to \spec{A}
  \end{equation*}
  は$(\varphi^{a})^{\#}(\spec{A})=\varphi$を満たすスキームの射である.
}{
  $X=\spec{B},Y=\spec{A}$と置く.任意の$g\in A$に対して
  \begin{equation*}
    (\varphi^{a})^{-1}(D(g))=D(\varphi(g))
  \end{equation*}
  が成り立ち,$\varphi$から誘導される環の射
  \begin{equation*}
    (\varphi^{a})^{\#}(D(g)):\mathcal{O}_{Y}(D(g)) = A_{g} \to B_{\varphi(g)} = (\varphi^{a})_{*}\mathcal{O}_{X}(D(g))
  \end{equation*}
  これは制限写像と可換(compatibleという意味で)になる.よって層の射
  \begin{equation*}
    (\varphi^{a})^{\#}:\mathcal{O}_{Y} \to \varphi^{a}_{*}\mathcal{O}_{X}
  \end{equation*}
  に拡張できる.
  %開基?(or 基本開集合族)で決まっていれば拡張できることを証明する.
  更に,任意の$\mathfrak{q} \in X$に対して$\varphi$から誘導される環の射
  \begin{equation*}
    A_{\varphi^{a}(\mathfrak{q})} \to B_{\mathfrak{q}}
  \end{equation*}
  は局所射で$(\varphi^{a})^{\#}_{x}$に一致する.よって$(\varphi^{a},(\varphi^{a})^{\#})$は
  局所環付き空間の射になる.構成により$(\varphi^{a})^{\#}(Y)=\varphi$を満たす.
}