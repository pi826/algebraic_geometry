
\Section{Zariski Topology}
$\spec{A}$を幾何的な対象に昇華するために,位相を導入しよう.まず,環$A$のイデアル$I$に対して
\begin{align*}
  V(I) &= \{\mathfrak{p} \in \spec{A}\mid I\subset \mathfrak{p}\}\\
  D(I) &= \spec{A}\mysetminus V(I) = \{\mathfrak{p} \in \spec{A}\mid I\not\subset \mathfrak{p}\}
\end{align*}
更に,$f\in A$に対して
\begin{align*}
  V(f) &= \{\mathfrak{p} \in \spec{A}\mid Af \subset \mathfrak{p}\}\\
  D(f) &= \spec{A}\mysetminus V(Af) = \{\mathfrak{p} \in \spec{A}\mid Af \not\subset \mathfrak{p}\}
\end{align*}
と定義する.
また,$Af\subset \mathfrak{p}$より$af\in Af$は$af\in \mathfrak{p}$なので,$a=1$とすれば$f\in \mathfrak{p}$がわかり,
イデアルの定義より,
\begin{align*}
  V(f) &= \{\mathfrak{p} \in \spec{A} \mid f \in \mathfrak{p}\}\\
  D(f) &= \{\mathfrak{p} \in \spec{A} \mid f \notin \mathfrak{p}\}
\end{align*}
がわかる.次に$\{D(f)\}_{f\in A}$を開集合族とする位相が定まることを示そう.

\Proposition{
  ああああ
}{}