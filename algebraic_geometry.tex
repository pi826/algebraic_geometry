\documentclass[dvipdfmx,a4paper,11pt]{jsarticle}


% 数式
\usepackage{amsmath,amsfonts}
\usepackage{bm}
\usepackage{amsthm}
\usepackage{amssymb}
\usepackage{tcolorbox}
% 画像
% \usepackage[dvipdfmx]{graphicx,color}

\usepackage[%
dvipdfmx, %欧文ではコメントアウトする
setpagesize=false,
bookmarks=true,
bookmarksdepth=tocdepth,
bookmarksnumbered=true,
colorlinks=true,
linkcolor=blue,
citecolor=blue,
urlcolor=blue,
pdftitle={},
pdfsubject={},
pdfauthor={},
pdfkeywords={}
]{hyperref}

\usepackage{pxjahyper}
\usepackage{tikz}
\usetikzlibrary{intersections,calc,arrows.meta}
\usepackage[truedimen,top=25truemm,bottom=30truemm,hmargin=25truemm]{geometry}
\usepackage{calc}


\begin{document}
\theoremstyle{plain}
\newtheorem{thm}{Theorem}[section]
\newtheorem{lem}[thm]{Lemma}
\newtheorem{prop}[thm]{Proposition}
\newtheorem{cor}[thm]{Corollary}
\newtheorem{conj}[thm]{Conjecture}

\theoremstyle{definition}
\newtheorem{ass}[thm]{Assumption}
\newtheorem{dfn}[thm]{Definition}

\theoremstyle{remark}
\newtheorem{rem}[thm]{Remark}

\theoremstyle{plain}
\newtheorem*{thm*}{Theorem}
\newtheorem*{lem*}{Lemma}
\newtheorem*{prop*}{Proposition}
\newtheorem*{cor*}{Corollary}
\newtheorem*{conj*}{Conjecture}

\theoremstyle{definition}
\newtheorem*{ass*}{Assumption}
\newtheorem*{dfn*}{Definition}

\theoremstyle{remark}
\newtheorem*{Proof}{Proof}

\numberwithin{equation}{section}

\theoremstyle{remark}
\newtheorem*{rem*}{Remark}

% \setlength{\topmargin}{-20truemm}
% \setlength{\headheight}{0pt}
% \setlength{\headsep}{0pt}
\setlength{\footskip}{20truemm}


% \setlength{\@tempdima}{1pt*\ratio{\dimexpr\textheight/\@lines}{\baselineskip}}
% \renewcommand{\baselinestretch}{\strip@pt\@tempdima}\selectfont

\makeatletter
\newcount\@chars\newcount\@lines
\@chars=40                      % 1行の文字数
\@lines=40                      % 1ページの行数

\newdimen\@kanjiskip
\@kanjiskip=\dimexpr(\textwidth-1zw*\@chars)/\numexpr\@chars-1
\newdimen\@@kanjiskip
\@@kanjiskip=\dimexpr\@kanjiskip/10

\baselineskip=\dimexpr\textheight/\@lines
\kanjiskip=\@kanjiskip plus \@@kanjiskip minus \@@kanjiskip
\parindent=\dimexpr 1zw+2truept
\parindent=\dimexpr\parindent+\@kanjiskip
\makeatother

% ↑は貼り付けただけなのでわからない。

\title{代数幾何}
\date{}
\author{Fefr}
\maketitle
\tableofcontents
\clearpage

%--------本文---------%
\section{代数多様体}

\subsection{代数的集合}

代数幾何学は代数方程式で定められる図形の幾何学である.一番素朴な形では,体$k$の元を係数とする連立方程式
\begin{equation}
  f_{\alpha}(x_{1},x_{2},\cdots,x_{n})=0\quad \alpha = 1,2,\cdots,l
\end{equation}
の解全体を幾何学的に考察することに他ならない.\\
しばらく,体$k$を代数的閉体と仮定して話を進める.体$k$の元の$n$個の組$(a_{1},a_{2},\cdots,a_{n})$の全体を
$k^n$と記し,体$k$上の$n$次元\textgt{アフィン空間}(affine space)と呼ぶ、$k^n$は体$k$上の$n$次元ベクトル空間の構造を持つ.\\
さて,連立方程式(1.1)の体$k$での解の全体を$V(f_{1},f_{2},\cdots,f_{l})$としるし,連立方程式(1.1)が定める\textgt{代数的集合}(algebraic set)または
\textgt{アフィン代数的集合}(affine algebraic set)と呼ぶ,すなわち
\begin{equation*}
  V(f_{1},f_{2},\cdots,f_{l})=\{(a_{1},a_{2},\cdots,a_{n})\in k^{n}\ |\ f_{\alpha}(a_{1},a_{2},\cdots,a_{n})=0,\ \alpha = 1,2,\cdots,l\}
\end{equation*}
一方,$\ f_{1},f_{2},\cdots,f_{l}$より生成される$n$変数多項式環$k[x_{1},x_{2},\cdots,x_{n}]$のイデアル$(f_{1},f_{2},\cdots,f_{l})$の任意の元$f(x_{1},x_{2},\cdots,x_{n})$に対して,
$(a_{1},a_{2},\cdots,a_{n})\in V(f_{1},f_{2},\cdots,f_{l})$であれば,
\begin{equation*}
  f(a_{1},a_{2},\cdots,a_{n})=0
\end{equation*}
が成り立つ.\\
多項式環$k[x_{1},x_{2},\cdots,x_{n}]$のイデアル$I$に対して
\begin{equation*}
  V(I)=\{(a_{1},a_{2},\cdots,a_{n})\in k^{n}\ |\ \forall f\in I:f(a_{1},a_{2},\cdots,a_{n})=0\}
\end{equation*}
と定義し,$V(I)$をイデアル$I$が定める代数的集合またはアフィン代数的集合という.すると,次の補題が成り立つ.
\begin{tcolorbox}[title = 補題1.1,upperbox = visible]
  $I=(f_{1},f_{2},\cdots,f_{l})$のとき
  \begin{equation*}
    V(I)=V(f_{1},f_{2},\cdots,f_{l})
  \end{equation*}
  \tcblower
  \textgt{証明}\\
  $V(f_{1},f_{2},\cdots,f_{l})\subset V(I)$は上で示した.逆に$(a_{1},a_{2},\cdots,a_{n})\in V(I)$であれば,\\
  $f_{\alpha}\in I(\alpha = 1,2,\cdots,l)$より
  \begin{equation*}
    f_{\alpha}(a_{1},a_{2},\cdots,a_{n})=0
  \end{equation*}
  が成り立ち,$V(I)\subset V(f_{1},f_{2},\cdots,f_{l})$がわかる.
\end{tcolorbox}
よって,今後は連立方程式(1.1)の代わりにイデアル$I$から定まる代数的集合$V(I)$を考えることにする.
\clearpage
ところで,零イデアル$(0)$に対して$V((0))=k^n$であるので$k^n$もアフィン代数的集合と考えることができる.
そこで,以下体$k$上の$n$次元アフィン空間を$\mathbb{A}^{n}_{k}$と記すことにする.また,体$k$が文脈的に明らかであれば$\mathbb{A}^n$と記すことが多い.\\
さて,このようにイデアルが定めるアフィン代数的集合を考えても,実際は連立方程式を考えることと本質的に同じであることは、\textgt{Hilbertの基底定理}(Hilbert's basis theorem)が保証する.
\begin{tcolorbox}[title = 定理1.2(Hilbertの基底定理)]
  多項式環$k[x_{1},x_{2},\cdots,x_{n}]$のイデアルは有限生成である.すなわち,イデアル$I$は
  \begin{equation*}
    I=(f_{1},f_{2},\cdots,f_{n})
  \end{equation*}
  と表すことができる.
\end{tcolorbox}
\begin{tcolorbox}[title = 命題1.3]
  体$k$上の多項式環$k[x_{1},x_{2},\cdots,x_{n}]$のイデアル$I,J,I_{\lambda}(\lambda\in \Lambda)$($\Lambda$は無限集合でもよい)に関して以下が成り立つ.
  \begin{itemize}
    \item[(1)] $V(I)\cup V(J)=V(I\cap J)$
    \item[(2)] $\bigcap_{\lambda\in \Lambda} V(I_{\lambda})=V(\sum_{\lambda\in \Lambda}I_{\lambda})$ 
    \item[(3)] $\sqrt{I}\subset \sqrt{J}$ならば$V(I)\supset V(J)$
  \end{itemize}
\end{tcolorbox}
\begin{tcolorbox}[title = 証明(1)]
  (1)$I\subset J$であれば$V(I)\supset V(J)$であることに注意すると,
  \begin{equation*}
    V(I\cap J) \supset V(I),\quad V(I\cap J)\supset V(J)
  \end{equation*}
  がわかる.したがって,
  \begin{equation*}
    V(I)\cup V(J)\subset V(I\cap J)
  \end{equation*}
  が成り立つ.逆に$(a_{1},a_{2},\cdots,a_{n})\in V(I\cap J)$を考える.もし$(a_{1},a_{2},\cdots,a_{n})\notin V(I)$であれば,
  \begin{equation*}
    f(a_{1},a_{2},\cdots,a_{n})\neq0
  \end{equation*}
  を満たす$f\in I$がある.このとき,任意の元$g(x_{1},x_{2},\cdots,x_{n})\in J$に対して$h=fg\in I\cap J$であるので,
  \begin{equation*}
    h(a_{1},a_{2},\cdots,a_{n})=f(a_{1},a_{2},\cdots,a_{n})g(a_{1},a_{2},\cdots,a_{n})=0
  \end{equation*}
  であり,
  \begin{equation*}
    g(a_{1},a_{2},\cdots,a_{n})=0
  \end{equation*}
  が成り立つ.よって$(a_{1},a_{2},\cdots,a_{n})\in V(J)$である.したがって
  \begin{equation*}
    V(I\cap J)\subset V(I)\cup V(J)
  \end{equation*}
  が成り立ち(1)がわかる.
\end{tcolorbox}
\begin{tcolorbox}[title = 証明(2)]
  $I_{\mu}\subset \sum_{\lambda\in \Lambda}I_{\lambda}(\mu \in \Lambda)$であるので
  \begin{equation*}
    V(I_{\mu})\supset V(\sum_{\lambda\in \Lambda}I_{\lambda})
  \end{equation*}
  が成り立ち,したがって
  \begin{equation*}
    \bigcap_{\lambda \in \Lambda}V(\lambda)\supset V(\sum_{\lambda\in \Lambda}I_{\lambda})
  \end{equation*}
  が成り立つことがわかる.各$\lambda$に対して
  \begin{equation*}
    I_{\lambda}=(h_{\lambda 1},h_{\lambda 2},\cdots,h_{\lambda m_{\lambda}})
  \end{equation*}
  とすると,$(a_{1},a_{2},\cdots,a_{n})\in \bigcap_{\lambda \in \Lambda}V(I_{\lambda})$であれば
  \begin{equation*}
    h_{\lambda j}(a_{1},a_{2},\cdots,a_{n})=0,\quad j=1,2,\cdots,m_{\lambda}
  \end{equation*}
  が成り立つ.\\
  一方$\{h_{\lambda j}\}_{\lambda \in \Lambda,\ 1\leq j\leq m_{\lambda}}$はイデアル$\sum_{\lambda \in \Lambda}I_{\lambda}$を生成するので,
  $(a_{1},a_{2},\cdots,a_{n})\in V(\sum_{\lambda\in \Lambda}I_{\lambda})$が成り立つ.
\end{tcolorbox}
\begin{tcolorbox}[title = 証明(3)]
  $V(\sqrt{I})=V(I)$を示せば十分である.$\sqrt{I}\supset I$であるので
  \begin{equation*}
    V(\sqrt{I})\subset V(I)
  \end{equation*}
  が成り立つ.一方$f\in \sqrt{I}$であれば$f^{m}\in I$となる正整数$m$が存在する.\\
  $(a_{1},a_{2},\cdots,a_{n})\in V(I)$であれば
  \begin{equation*}
    (f(a_{1},a_{2},\cdots,a_{n}))^{m}=0
  \end{equation*}
  となり,したがって
  \begin{equation*}
    f(a_{1},a_{2},\cdots,a_{n})=0
  \end{equation*}
  が成り立つ.これは$(a_{1},a_{2},\cdots,a_{n})\in V(\sqrt{I})$を意味する.よって
  \begin{equation*}
    V(\sqrt{I})\supset V(I)
  \end{equation*}
\end{tcolorbox}
\begin{tcolorbox}[title = 系1.4]
  $k[x_{1},x_{2},\cdots,x_{n}]$の有限個のイデアル$I_{1},I_{2},\cdots,I_{s}$に関して
  \begin{equation*}
    \bigcup_{j=1}^{s}V(I_{j})=V(\bigcap_{j=1}^{s}I_{j})
  \end{equation*}
\end{tcolorbox}

\subsection{Hilbertの零点定理}

$n$次元アフィン空間$\mathbb{A}^{n}_{k}$内の代数的集合が幾何的な意味を持つには$V(I)\neq \varnothing$である必要がある.
次の\textgt{弱い形のHilbertの零点定理}(week Hilbert's Nullstellensatz)と呼ばれる定理はこれを保証する.
\begin{tcolorbox}[title = 定理1.5,upperbox = visible]
  代数的閉体$k$上の多項式環$k[x_{1},x_{2},\cdots,x_{n}]$のイデアル$I$が単位元を含まない,すなわち
  $I\neq k[x_{1},x_{2},\cdots,x_{n}]$であれば
  \begin{equation*}
    V(I)\neq \varnothing
  \end{equation*}
  \tcblower
  \textgt{証明}\\
  $I\neq k[x_{1},x_{2},\cdots,x_{n}]$であればイデアル$I$は必ずある極大イデアル$\mathfrak{m}$に含まれる.
  $I\subset \mathfrak{m}$より$V(I)\supset V(\mathfrak{m})$である.
  したがって$V(\mathfrak{m})$が空でなければ十分である.ゆえに$I=\mathfrak{m}$の時を考える.
  極大イデアル$\mathfrak{m}$に対して剰余環$k[x_{1},x_{2},\cdots,x_{n}]/\mathfrak{m}$は$k$を含む体である.
  $k$は代数的閉体であるので,下の補題1.7より$k[x_{1},x_{2},\cdots,x_{n}]/\mathfrak{m}=k$でなければならない.
  よって$x_{j}$の$\mathfrak{m}$に関する剰余類$\overline{x_{j}}$は$k$の元$a_{j}$を定める.
  すなわち$x_{j}-a_{j}\equiv 0\ (\! \bmod \mathfrak{m}\, )$であるので,$x_{j}-a_{j}\in \mathfrak{m}$である.
  これは$(x_{1}-a_{1},x_{2}-a_{2},\cdots,x_{n}-a_{n})\subset \mathfrak{m}$を意味するが$(x_{1}-a_{1},x_{2}-a_{2},\cdots,x_{n}-a_{n})$は極大イデアルなので
  \begin{equation*}
    (x_{1}-a_{1},x_{2}-a_{2},\cdots,x_{n}-a_{n})=\mathfrak{m}
  \end{equation*}
  となり,
  \begin{equation*}
    V(\mathfrak{m})=\{(a_{1},a_{2},\cdots,a_{n})\}
  \end{equation*}
  がわかる.
\end{tcolorbox}
\begin{tcolorbox}[title = 系1.6]
  代数的閉体$k$上の多項式環$k[x_{1},x_{2},\cdots,x_{n}]$の極大イデアルは
  \begin{equation*}
    (x_{1}-a_{1},x_{2}-a_{2},\cdots,x_{n}-a_{n}),\quad a_{j}\in k,\quad j = 1,2,\cdots,n
  \end{equation*}
  の形をしている.
\end{tcolorbox}
定理1.5,系1.6どちらも体$k$が代数的閉体であることが本質的である.$\mathbf{R}$上では成り立たない.

\begin{tcolorbox}[title = 補題1.7,upperbox = visible]
  体$k$上有限生成の整域$R$が体であれば,$R$の各元は$k$上代数的である.
  \tcblower
  \textgt{証明}\\
  仮定より
  \begin{equation}
    R=k[z_{1},z_{2},\cdots,z_{m}]
  \end{equation}
  となる$z_{1},z_{2},\cdots,z_{m}\in R$が存在する.$z_{1},z_{2},\cdots,z_{m}$が$k$上代数的であることを示せばよい.
  このことを生成元の個数$m$に関する帰納法で示す.$m=1$のときは,もし$z_{1}$が$k$上代数的でなければ$k$上超越的であり,
  $k[z_{1}]$は$k$上の多項式環と同型であり体ではない.これは$R=k[z_{1}]$が体であるという仮定に反する.よって$z_{1}$は$k$上代数的である.
  次に$m\geq 2$と仮定する.$z_{1}\in R$に対して$k$の拡大体$k(z_{1})$は$R$の部分体である.よって
  \begin{equation*}
    R=k(z_{1})[z_{2},z_{3},\cdots,z_{m}]
  \end{equation*}
  
\end{tcolorbox}




\end{document}