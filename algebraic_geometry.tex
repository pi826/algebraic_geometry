\documentclass[dvipdfmx,a4paper,11pt]{jsbook}


% 数式
\usepackage{amsmath,amsfonts}
\usepackage{mathtools}
\usepackage{xcolor}
\usepackage{xcoffins,calc}
\usepackage{bm}
\usepackage{amsthm}
\usepackage{amssymb}
\usepackage{pgf}
\usepackage{tcolorbox}
\usepackage{titlesec}
\usepackage{ifthen}
\usepackage{mathrsfs}

% 画像
% \usepackage[dvipdfmx]{graphicx,color}

\usepackage[%
dvipdfmx, %欧文ではコメントアウトする
setpagesize=false,
bookmarks=true,
bookmarksdepth=tocdepth,
bookmarksnumbered=true,
colorlinks=true,
linkcolor=blue,
citecolor=blue,
urlcolor=blue,
pdftitle={},
pdfsubject={},
pdfauthor={},
pdfkeywords={}
]{hyperref}

\usepackage{pxjahyper}
\usepackage{tikz}
\usetikzlibrary{intersections,calc,arrows.meta}
\usepackage[truedimen,top=25truemm,bottom=30truemm,hmargin=25truemm]{geometry}
\usepackage{calc}
\usepackage{fancyhdr}
\pagestyle{fancy}


\pagestyle{fancy}
\fancyhf{} % 全てのヘッダーとフッターをクリア
\fancyhead[LE,RO]{\thepage} % 左側の偶数ページと右側の奇数ページにページ番号を表示
\fancyhead[RE]{\leftmark} % 右側の偶数ページに章名を表示
\fancyhead[LO]{\rightmark} % 左側の奇数ページに節名を表示
\renewcommand{\headrulewidth}{0.4pt} % ヘッダーの線の太さを設定
\renewcommand{\footrulewidth}{0pt} % フッターの線の太さを定

% \makeatletter
% \let\old@rule\@rule
% \def\@rule[#1]#2#3{\textcolor{blue}{\old@rule[#1]{#2}{#3}}}
% \makeatother
% \newtcolorbox{mybox}[2][]{enhanced,skin=enhancedlast jigsaw,
% attach boxed title to top left={xshift=-4mm,yshift=-0.5mm},
% fonttitle=\bfseries\sffamily,varwidth boxed title=0.7\linewidth,
% colbacktitle=blue!45!white,colframe=red!50!black,
% interior style={top color=blue!10!white,bottom color=red!10!white},
% boxed title style={empty,arc=0pt,outer arc=0pt,boxrule=0pt},
% underlay boxed title={
% \fill[blue!45!white] (title.north west) -- (title.north east)
% -- +(\tcboxedtitleheight-1mm,-\tcboxedtitleheight+1mm)
% -- ([xshift=4mm,yshift=0.5mm]frame.north east) -- +(0mm,-1mm)
% -- (title.south west) -- cycle;
% \fill[blue!45!white!50!black] ([yshift=-0.5mm]frame.north west)
% -- +(-0.4,0) -- +(0,-0.3) -- cycle;
% \fill[blue!45!white!50!black] ([yshift=-0.5mm]frame.north east)
% -- +(0,-0.3) -- +(0.4,0) -- cycle; },
% title={#2},#1}

% chapter 
\titleformat{\chapter}[block]
{}{}{0pt}{
  \fontsize{27pt}{30pt}\selectfont\filleft
}[
  \ifthenelse{\value{chapter}=0}{\hrule}{\titleline{
  \hspace{250pt}
  \begin{tikzpicture}
    \draw [line width = 0.4pt] (0,0) -- (6.5cm,0);
    % \draw [line width = 0.4pt] (6.5cm,1cm) -- (6.5cm,-1cm);
  \end{tikzpicture}}}
  \Large{\filleft \ifthenelse{\value{chapter}=0}{}{第 \thechapter 章}}
]

\makeatletter
\def\Section{\@ifstar{\@Section[2pt]}{\@Section[\z@]}}
%
\def\@Section[#1]#2{\ifdim #1<1pt\refstepcounter{section}\fi%
\section*{\nopagebreak[4] \vskip.5pc%
\ifdim #1<1pt%
\addcontentsline{toc}{section}{\protect\numberline{\thesection}#2}%
\quad \textbf{\thesection~} \fi \raisebox{-1.5ex}[0pt][0pt]{\color[rgb]{0.6,0.8,1}{\rule{3mm}{2em}}} #2\nopagebreak[4]%
\vskip.25pc \hrule\@height 1pt\nopagebreak[4] \vskip1pc}}
\makeatother

% % 定理環境の設定
\tcbuselibrary{skins, breakable, theorems}
\usepackage{cleveref}
% \newcommand{\kara}{}
% \newtcolorbox[auto counter, number within = section, crefname = {Def.}{Defs.}]{definition}[3][]{enhanced, breakable = true, fonttitle = \bfseries,title = Def.~\thetcbcounter~\if #2\kara \else (#2) \fi, #1, label = the:#3, boxrule=0pt, frame hidden, borderline west={4pt}{0pt}{green!75!black},
% colback=green!10!white,sharp corners}



% \newtheorem{theorem}{Theorem}[section]
% \tcolorboxenvironment{theorem}{
%   colback=blue!5!white,
%   boxrule=0pt,
%   boxsep=1pt,
%   left=2pt,right=2pt,top=2pt,bottom=2pt,
%   oversize=2pt,
%   sharp corners,
%   before skip=\topsep,
%   after skip=\topsep,
% }




\renewcommand{\qedsymbol}{$\blacksquare$}
\newcommand{\Claim}[1]{\underline{\textbf{Claim#1.}}}

\newcommand{\kara}{}%
\newcounter{totalcounter}
\renewcommand{\thetotalcounter}{\thechapter.\thesection.\arabic{totalcounter}}
\newcounter{defcounter}
\renewcommand{\thedefcounter}{\thechapter.\thesection.\arabic{defcounter}}

\NewTotalTColorBox[use counter = defcounter, number within = section]{\Definition}{ +m }{ 
    notitle,
    colback=green!5!white,
    frame hidden,
    boxrule=0pt,
    enhanced,
    sharp corners,
    borderline west={4pt}{0pt}{green!50!black},
    breakable = true,
    label = {Def:\thedefcounter},
}{
    \textcolor{green!50!black}{
        \sffamily
        \textbf{Definition~\thetcbcounter.}%k
    }%
    #1
}

\NewTotalTColorBox{\Remark}{ +m +m }{ 
    notitle,
    colback=yellow!5!white,
    colbacklower=white,
    frame hidden,
    boxrule=0pt,
    bicolor,
    sharp corners,
    borderline west={4pt}{0pt}{yellow!50!black},
    %fontupper=\sffamily,
    breakable = true,
    label = {Rem:\thetotalcounter},
}{
    \textcolor{yellow!50!black}{
        \sffamily
        \textbf{Remark~.}%
    }%
    #1
    \if #2\kara \else 
    \tcblower%
    \textcolor{yellow!50!black}{
        \sffamily
        \textbf{Proof.}%
    }%
    #2
    \qedsymbol
    \fi
}

\NewTotalTColorBox[use counter = totalcounter, number within = section]{\Lemma}{ +m +m }{ 
    notitle,
    colback=orange!5!white,
    colbacklower=white,
    frame hidden,
    boxrule=0pt,
    bicolor,
    sharp corners,
    borderline west={4pt}{0pt}{orange!50!black},
    fontupper=\sffamily,
    breakable = true,
    label = {Lem:\thetotalcounter},
}{
    \textcolor{orange!50!black}{
        \sffamily
        \textbf{Lemma~\thetcbcounter.}%
    }%
    #1
    \if #2\kara \else 
    \tcblower%
    \textcolor{orange!50!black}{
        \sffamily
        \textbf{Proof.}%
    }%
    #2
    \qedsymbol
    \fi
}


\NewTotalTColorBox[use counter = totalcounter, number within = section]{\Theorem}{ +m +m }{ 
    notitle,
    colback=blue!5!white,
    colbacklower=white,
    frame hidden,
    boxrule=0pt,
    bicolor,
    sharp corners,
    borderline west={4pt}{0pt}{blue!50!black},
    fontupper=\sffamily,
    breakable = true,
    label = {Thm:\thetotalcounter},
}{
    \textcolor{blue!50!black}{
        \sffamily
        \textbf{Theorem~\thetcbcounter.}%
    }%
    #1
    \if #2\kara \else 
    \tcblower%
    \textcolor{blue!50!black}{
        \sffamily
        \textbf{Proof.}%
    }%
    #2
    \qedsymbol
    \fi
}

\NewTotalTColorBox[use counter = totalcounter, number within = section]{\Example}{ +m +m }{ 
    notitle,
    colback=cyan!5!white,
    colbacklower=white,
    frame hidden,
    boxrule=0pt,
    bicolor,
    sharp corners,
    borderline west={4pt}{0pt}{cyan!50!black},
    %fontupper=\sffamily,
    breakable = true,
    label = {Ex:\thetotalcounter},
}{
    \textcolor{cyan!50!black}{
        \sffamily
        \textbf{Example~\thetcbcounter.}%
    }%
    #1
    \if #2\kara \else
    \tcblower%
    \textcolor{cyan!50!black}{
        \sffamily
        \textbf{Proof.}%
    }%
    #2
    \qedsymbol
    \fi
}

\NewTotalTColorBox[use counter = totalcounter, number within = section]{\Proposition}{ +m +m }{ 
    notitle,
    colback=red!5!white,
    colbacklower=white,
    frame hidden,
    boxrule=0pt,
    bicolor,
    sharp corners,
    borderline west={4pt}{0pt}{red!50!black},
    fontupper=\sffamily,
    breakable = true,
    label = {Prop:\thetotalcounter},
}{
    \textcolor{red!50!black}{
        \sffamily
        \textbf{Proposition~\thetcbcounter.}%
    }%
    #1
    \if #2\kara \else 
    \tcblower%
    \textcolor{red!50!black}{
        \sffamily
        \textbf{Proof.}%
    }%
    #2
    \qedsymbol
    \fi
}
\NewTotalTColorBox[use counter = totalcounter, number within = section]{\Corollary}{ +m +m }{ 
    notitle,
    colback=darkgray!5!white,
    colbacklower=white,
    frame hidden,
    boxrule=0pt,
    bicolor,
    sharp corners,
    borderline west={4pt}{0pt}{darkgray!50!black},
    fontupper=\sffamily,
    breakable = true,
    label = {Cor:\thetotalcounter},
}{
    \textcolor{darkgray!50!black}{
        \sffamily
        \textbf{Corollary~\thetcbcounter.}%
    }%
    #1
    \if #2\kara \else 
    \tcblower%
    \textcolor{darkgray!50!black}{
        \sffamily
        \textbf{Proof.}%
    }%
    #2
    \qedsymbol
    \fi
}

% \newtcbtheorem[use counter from = theorem]{mythm}{Theorem}%
% {
%   colback=white, % ボックス内の背景色
%   colframe=black, % フレームの色
%   fonttitle=\bfseries, % タイトルのフォント
% }{thm}

% % 証明環境の設定
% \newtcbtheorem[use counter from = theorem]{mypf}{Proof}%
% {
%   colback=white, 
%   colframe=black, 
%   fonttitle=\bfseries
% }{prf}

% % 定義環境の設定
% \newtcbtheorem[use counter from = theorem]{mydef}{Definition}%
% {
%   colback=white, 
%   colframe=black, 
%   fonttitle=\bfseries
% }{def}

\newtcolorbox{mybox}{
  enhanced,
  boxrule=0pt,frame hidden,
  borderline west={4pt}{0pt}{green!75!black},
  colback=green!10!white,
  sharp corners
}

\begin{document}

% \setlength{\topmargin}{-20truemm}
% \setlength{\headheight}{0pt}
% \setlength{\headsep}{0pt}
\setlength{\footskip}{20truemm}



% \setlength{\@tempdima}{1pt*\ratio{\dimexpr\textheight/\@lines}{\baselineskip}}
% \renewcommand{\baselinestretch}{\strip@pt\@tempdima}\selectfont

\makeatletter
\newcount\@chars\newcount\@lines
\@chars=40                      % 1行の文字数
\@lines=40                      % 1ページの行数

\newdimen\@kanjiskip
\@kanjiskip=\dimexpr(\textwidth-1zw*\@chars)/\numexpr\@chars-1
\newdimen\@@kanjiskip
\@@kanjiskip=\dimexpr\@kanjiskip/10

\baselineskip=\dimexpr\textheight/\@lines
\kanjiskip=\@kanjiskip plus \@@kanjiskip minus \@@kanjiskip
\parindent=\dimexpr 1zw+2truept
\parindent=\dimexpr\parindent+\@kanjiskip
\makeatother

% ↑は貼り付けただけなのでわからない。



\title{代数幾何まとめノート}
\date{\today}
\author{Fefr}
\maketitle




\tableofcontents 
\clearpage
%--------------------本文--------------------%

\chapter{Scheme}
\Section{Zariski Topology}
atodekakuyo
\Section{Algebraic Sets}
atodekakuyo
\Section{Sheaves}
\Definition{
  $X$を位相空間とする。$X$上の(アーベル群の)\textbf{前層}(presheaf)$\, \mathcal{F}$とは
  次のデータ
  \begin{center}
    \begin{itemize}
      \item[---] $U$を任意の$X$の開集合に対して$\mathcal{F}(U)$はアーベル群。
      \item[---] 制限写像(restriction map)と言われる群準同型
      $\rho_{U,V}:\mathcal{F}(U) \to \mathcal{F}(V)$が任意の開集合$V\subset U$に対して存在する。 
    \end{itemize}
  \end{center}
  そして次の条件を満たす。
  \begin{itemize}
    \item[(1)] $\mathcal{F}(\varnothing) = 0$
    \item[(2)] $\rho_{U,U} = \text{id}_{\mathcal{F}(U)}$
    \item[(3)] 任意の開集合$W\subset V \subset U$に対して$\rho_{U,W}=\rho_{V,W}\circ \rho_{U,V}$となる。  
  \end{itemize}
}
$s\in \mathcal{F}(U)$を$U$上の$\mathcal{F}$の\textbf{切断}(section)という。
また、$\rho_{U,V}(s)\in \mathcal{F}(V)$を$s|_{V}$と書いて$s$の$V$への制限という。

\Definition{
  前層$\mathcal{F}$が層(sheaf)とは次の条件を満たすことをいう。
  \begin{itemize}
    \item[(4)] (Uniqueness) $U$を$X$の開集合とし$\{U_{i}\}_{i}$をその開被覆とする。
                $s\in \mathcal{F}(U)$が任意の$i$に対して$s|_{U_{i}}=0$ならば$s=0$
    \item[(5)] (Glueing local sections) 上の状況で、$s_i \in \mathcal{F}(U_i)$が
                $s_{i}|_{U_{i} \cap U_{j}} = s_{j}|_{U_{i} \cap U_{j}}$を満たすならば、$s|_{U_{i}} = s_{i}$を満たす$s\in \mathcal{F}(U)$が存在する。
  \end{itemize}
}
\Remark{
  $\mathcal{B}$を位相空間$X$の開基で有限交叉で閉じているものとする。(つまり任意の$U,V\in \mathcal{B}$に対して$U\cap V \in \mathcal{B}$.\ e.g. $\text{Spec}\, A$の開基$\{D(f)\}_f$)
  このとき$\mathcal{B}$-前層($\mathcal{B}\text{-presheaf}$)\\$\mathcal{F}_0$とは
  \begin{itemize}
    \item[---] $U\in \mathcal{B}$に対して$\mathcal{F}_0(U)$はアーベル群。
    \item[---] $V\subset U \in \mathcal{B}$に対して群準同型$\rho_{U,V}:\mathcal{F}_0(U) \to \mathcal{F}_0(V)$が定まる。
  \end{itemize}
  としたもの。\\
  $\mathcal{B}$-層($\mathcal{B}\text{-sheaf}$)$\mathcal{F}_0$から$X$上の層$\mathcal{F}$を作ることができる。\\
  位相空間$X$の任意の開集合$U$をとり、$\{U_{i}\}_{i}$をその開被覆とする。($U_{i} \in \mathcal{B}$)
  \begin{equation*}
    \mathcal{F}(U):=\left\{(s_{i})_{i} \in \prod_{i}\mathcal{F}_0(U_{i})\ \Biggl|\ 任意のi,jに対してs_{i}|_{U_{i} \cap U_{j}} = s_{j}|_{U_{i} \cap U_{j}}\right\}
  \end{equation*}
  と定義する。するとこれは開被覆によらない。実際$\mathcal{F}(U)_{U_i}$を開被覆$\{U_{i}\}_{i}$による$\mathcal{F}(U)$とし、$\{V_{j}\}_{j}$を別の開被覆とすると、
  $\{U_{i}\cap V_{j}\}_{i,j}$はこれら2つの細分である。$\mathcal{F}(U)_{U_i} \to \mathcal{F}(U)_{U_{i}\cap V_{j}}$なる群準同型を
  $(s_{i})_{i}\mapsto (s_{i}|_{U_{i}\cap V_{j}})_{i,j}$で定義できる。実際
  \begin{align*}
    s_{i} |_{U_{i} \cap V_{j}}\Bigl|_{(U_{i} \cap V_{j}) \cap (U_{i'} \cap V_{j'})} 
    &= s_{i} \Bigl|_{(U_{i} \cap V_{j}) \cap (U_{i'} \cap V_{j'})}\\
    &= s_{i} |_{U_{i}\cap U_{i'}}\Bigl|_{(U_{i} \cap V_{j}) \cap (U_{i'} \cap V_{j'})}\\
    &= s_{i'} |_{U_{i}\cap U_{i'}}\Bigl|_{(U_{i} \cap V_{j}) \cap (U_{i'} \cap V_{j'})} \quad (\because (s_{i})_{i} \in \mathcal{F}(U)_{U_{i}})\\
    &= s_{i'} \Bigl|_{(U_{i} \cap V_{j}) \cap (U_{i'} \cap V_{j'})}\\
    &= s_{i'} |_{U_{i'} \cap V_{j'}}\Bigl|_{(U_{i} \cap V_{j}) \cap (U_{i'} \cap V_{j'})}
  \end{align*}
  より$(s_{i}|_{U_{i}\cap V_{j}})_{i,j}\in \mathcal{F}(U)_{U_{i} \cap V_{j}}$\\
  また、$(s_{ij})_{ij}\in \mathcal{F}(U)_{U_{i}\cap V_{j}}$を取ると、
  $(s_{ij})_{ij} = (s_{i}|_{U_{i} \cap V_{j}})$と出来るので全射(?????)\\
  Kernelを計算すると
  \begin{align*}
    &s_{i}|_{U_{i}\cap V_{j}} = 0 \quad (\forall i,j)\\
    &s_{i}|_{U_{i}}=s_{i} = 0 \quad (\forall i) \quad (\because (4))
  \end{align*}
  よってKernelが自明なので単射。
  % $\mathcal{U}=\{U_{i}\}_{i}$を$X$の部分開集合族とする。$U=\cup_{i}U_{i},U_{ij}=U_{i}\cap U_{j}$
}{}
\end{document}