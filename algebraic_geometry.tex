\documentclass[dvipdfmx,a4paper,11pt]{jsarticle}


% 数式
\usepackage{amsmath,amsfonts}
\usepackage{bm}
\usepackage{amsthm}
\usepackage{amssymb}
\usepackage{tcolorbox}
% 画像
% \usepackage[dvipdfmx]{graphicx,color}

\usepackage[%
dvipdfmx, %欧文ではコメントアウトする
setpagesize=false,
bookmarks=true,
bookmarksdepth=tocdepth,
bookmarksnumbered=true,
colorlinks=true,
linkcolor=blue,
citecolor=blue,
urlcolor=blue,
pdftitle={},
pdfsubject={},
pdfauthor={},
pdfkeywords={}
]{hyperref}

\usepackage{pxjahyper}
\usepackage{tikz}
\usetikzlibrary{intersections,calc,arrows.meta}
\usepackage[truedimen,top=25truemm,bottom=30truemm,hmargin=25truemm]{geometry}
\usepackage{calc}


\begin{document}
\theoremstyle{plain}
\newtheorem{thm}{Theorem}[section]
\newtheorem{lem}[thm]{Lemma}
\newtheorem{prop}[thm]{Proposition}
\newtheorem{cor}[thm]{Corollary}
\newtheorem{conj}[thm]{Conjecture}

\theoremstyle{definition}
\newtheorem{ass}[thm]{Assumption}
\newtheorem{dfn}[thm]{Definition}

\theoremstyle{remark}
\newtheorem{rem}[thm]{Remark}

\theoremstyle{plain}
\newtheorem*{thm*}{Theorem}
\newtheorem*{lem*}{Lemma}
\newtheorem*{prop*}{Proposition}
\newtheorem*{cor*}{Corollary}
\newtheorem*{conj*}{Conjecture}

\theoremstyle{definition}
\newtheorem*{ass*}{Assumption}
\newtheorem*{dfn*}{Definition}

\theoremstyle{remark}
\newtheorem*{Proof}{Proof}

\numberwithin{equation}{section}

\theoremstyle{remark}
\newtheorem*{rem*}{Remark}

% \setlength{\topmargin}{-20truemm}
% \setlength{\headheight}{0pt}
% \setlength{\headsep}{0pt}
\setlength{\footskip}{20truemm}


% \setlength{\@tempdima}{1pt*\ratio{\dimexpr\textheight/\@lines}{\baselineskip}}
% \renewcommand{\baselinestretch}{\strip@pt\@tempdima}\selectfont

\makeatletter
\newcount\@chars\newcount\@lines
\@chars=40                      % 1行の文字数
\@lines=40                      % 1ページの行数

\newdimen\@kanjiskip
\@kanjiskip=\dimexpr(\textwidth-1zw*\@chars)/\numexpr\@chars-1
\newdimen\@@kanjiskip
\@@kanjiskip=\dimexpr\@kanjiskip/10

\baselineskip=\dimexpr\textheight/\@lines
\kanjiskip=\@kanjiskip plus \@@kanjiskip minus \@@kanjiskip
\parindent=\dimexpr 1zw+2truept
\parindent=\dimexpr\parindent+\@kanjiskip
\makeatother

% ↑は貼り付けただけなのでわからない。

\title{代数幾何}
\date{}
\author{Fefr}
\maketitle
\tableofcontents
\clearpage

%--------本文---------%
\section{代数多様体}

\subsection{代数的集合}

代数幾何学は代数方程式で定められる図形の幾何学である.一番素朴な形では,体$k$の元を係数とする連立方程式
\begin{equation}
  f_{\alpha}(x_{1},x_{2},\cdots,x_{n})=0\quad \alpha = 1,2,\cdots,l
\end{equation}
の解全体を幾何学的に考察することに他ならない.\\
しばらく,体$k$を代数的閉体と仮定して話を進める.体$k$の元の$n$個の組$(a_{1},a_{2},\cdots,a_{n})$の全体を
$k^n$と記し,体$k$上の$n$次元\textgt{アフィン空間}(affine space)と呼ぶ、$k^n$は体$k$上の$n$次元ベクトル空間の構造を持つ.\\
さて,連立方程式(1.1)の体$k$での解の全体を$V(f_{1},f_{2},\cdots,f_{l})$としるし,連立方程式(1.1)が定める\textgt{代数的集合}(algebraic set)または
\textgt{アフィン代数的集合}(affine algebraic set)と呼ぶ,すなわち
\begin{equation*}
  V(f_{1},f_{2},\cdots,f_{l})=\{(a_{1},a_{2},\cdots,a_{n})\in k^{n}\ |\ f_{\alpha}(a_{1},a_{2},\cdots,a_{n})=0,\ \alpha = 1,2,\cdots,l\}
\end{equation*}
一方,$\ f_{1},f_{2},\cdots,f_{l}$より生成される$n$変数多項式環$k[x_{1},x_{2},\cdots,x_{n}]$のイデアル$(f_{1},f_{2},\cdots,f_{l})$の任意の元$f(x_{1},x_{2},\cdots,x_{n})$に対して,
$(a_{1},a_{2},\cdots,a_{n})\in V(f_{1},f_{2},\cdots,f_{l})$であれば,
\begin{equation*}
  f(a_{1},a_{2},\cdots,a_{n})=0
\end{equation*}
が成り立つ.\\
多項式環$k[x_{1},x_{2},\cdots,x_{n}]$のイデアル$I$に対して
\begin{equation*}
  V(I)=\{(a_{1},a_{2},\cdots,a_{n})\in k^{n}\ |\ \forall f\in I:f(a_{1},a_{2},\cdots,a_{n})=0\}
\end{equation*}
と定義し,$V(I)$をイデアル$I$が定める代数的集合またはアフィン代数的集合という.すると,次の補題が成り立つ.
\begin{tcolorbox}[title = 補題1.1,upperbox = visible]
  $I=(f_{1},f_{2},\cdots,f_{l})$のとき
  \begin{equation*}
    V(I)=V(f_{1},f_{2},\cdots,f_{l})
  \end{equation*}
  \tcblower
  \textgt{証明}\\
  $V(f_{1},f_{2},\cdots,f_{l})\subset V(I)$は上で示した.逆に$(a_{1},a_{2},\cdots,a_{n})\in V(I)$であれば,\\
  $f_{\alpha}\in I(\alpha = 1,2,\cdots,l)$より
  \begin{equation*}
    f_{\alpha}(a_{1},a_{2},\cdots,a_{n})=0
  \end{equation*}
  が成り立ち,$V(I)\subset V(f_{1},f_{2},\cdots,f_{l})$がわかる.
\end{tcolorbox}
\end{document}