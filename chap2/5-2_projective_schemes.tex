
\subsection{Projective schemes}
\sectionmark{Projective schemes}
まず初めに次数環
\begin{equation*}
  A = \bigoplus_{n\in \mathbf{N}}A_{n}
\end{equation*}
を固定する.ここでイデアル$I \subset A$が斉次イデアルとは
\begin{equation*}
  I = \bigoplus_{n\in \mathbf{N}}(I\cap A_{n})
\end{equation*}
のときをいう.ここで
\begin{equation*}
  A/I = \left. \bigoplus_{n\in \mathbf{N}} A_{n}\right / \bigoplus_{n\in \mathbf{N}}(I\cap A_{n})
\end{equation*}
だが
$$
\begin{array}{rccc}
  \varphi \colon &\bigoplus_{n\in \mathbf{N}}A_{n}/(I\cap A_{n})                     &\longrightarrow& A/I                    \\
          & \rotatebox{90}{$\in$}&               & \rotatebox{90}{$\in$} \\
          & (x_{i} + I \cap A_{i})_{i}                    & \longmapsto   & (x_{i})_{i} + I
\end{array}
$$
とするとこれは全準同型で単射性は$(x_{i})_{i} + I = (y_{i})_{i} + I$とすると$(x_{i})_{i} - (y_{i})_{i}  = (x_{i} - y_{i})_{i}\in I$
と$x_{i} \in A_{i}$より$x_{i} - y_{i} \in I \cap A_{i}$でこれは単射であることを意味する.
よって,
\begin{equation*}
  A/I = \bigoplus_{n \in \mathbf{N}}A_{n}/(I \cap A_{n})
\end{equation*}
である.ここで$\proj{A}$を次のように定義しよう.
\begin{equation*}
  \proj{A}:= \left\{\mathfrak{p} \in \spec{A}\mid \mathfrak{p}は斉次イデアルでA_{+} \not \subset \mathfrak{p} \right\}
\end{equation*}
とおく.
%($A_{+}\not \subset \mathfrak{p}\Leftrightarrow A_{+}\cap \mathfrak{p}\not \subset A_{+}$であることに注意)
ただし
\begin{equation*}
  A_{+} := \bigoplus_{n > 0}A_{n}
\end{equation*}
である.あとで$\proj{A}$にスキームの構造が入ることを示そう.\\
任意の斉次イデアル$I\subset A$に対して
\begin{equation*}
  V_{+}(I) := \{\mathfrak{p} \in \proj{A}\mid I \subset \mathfrak{p}\}
\end{equation*}
と定義する.このとき
\begin{align}
  \bigcap_{\mu}V_{+}(I_{\mu}) &= V_{+}(\sum_{\mu}I_{\mu})\\
  V_{+}(I)\cup V_{+}(J) &= V_{+}(I\cap J)\\
  V_{+}(A) &= \varnothing\\
  V_{+}(0) &= \proj{A}
\end{align}
が成り立つ.実際(1.1)から示そう.
\begin{align*}
  I_{\lambda} \subset \sum_{\mu}I_{\mu}
\end{align*}
なので
\begin{equation*}
  V_{+}(\sum_{\mu}I_{\mu}) \subset V_{+}(I_{\lambda})
\end{equation*}
である.よって
\begin{equation*}
  V_{+}(\sum_{\mu}I_{\mu}) \subset \bigcap_{\mu}V_{+}(I_{\mu})
\end{equation*}
逆に$\mathfrak{p} \in \bigcap V_{+}(I_{\mu})$とすると任意の$\mu$に対して$I_{\mu} \subset \mathfrak{p}$なので
$\sum I_{\mu}\subset \mathfrak{p}$ が成り立ち逆の包含関係もわかる.\\
(1.2)は$\mathfrak{p} \in V_{+}(I)$なら$I \subset \mathfrak{p}$なので$I\cap J \subset \mathfrak{p}$
だから$V_{+}(I)\subset V_{+}(I\cap J)$で同様に$V_{+}(J) \subset V_{+}(I\cap J)$なので
\begin{equation*}
  V_{+}(I)\cup V_{+}(J) \subset V_{+}(I\cap J)
\end{equation*}
逆に$\mathfrak{p} \in V_{+}(I\cap J)$なら$I\cap J \subset \mathfrak{p}$で$I\not\subset \mathfrak{p}$なら$a\in I$かつ$a\notin \mathfrak{p}$なる元がある.しかし,任意の$b\in J$に対して
$ab \in I\cap J$なので$ab \in \mathfrak{p}$で今$a\notin \mathfrak{p}$なので$b\in \mathfrak{p}$
である.よって$J\subset \mathfrak{p}$なので$V_{+}(I\cap J)\subset V_{+}(J)$だから逆の包含関係もわかる.
残り二つは自明である.\\
$\spec$の場合と同様に$\proj{A}$にも$\{V_{+}(I)\}_{I}$を閉集合族とする位相を入れることにする.
この位相を同様にZariski位相ということにする.\\
$I$を$A$の任意のイデアルとすると,$I$に伴う斉次イデアル$I^{h} = \bigoplus(I\cap A_{n})$(ここで$h$乗ではなく単なる記号であることに注意)が定義できる.
\Lemma{
  $I,J$を次数環$A$のイデアルとする.このとき以下が成り立つ.
  \begin{itemize}
    \item[(1)] $I$が素イデアルならそれに伴う斉次イデアル$I^{h}$も素イデアル.
    \item[(2)] $I$と$J$が斉次イデアルとする.このとき
      \begin{equation*}
        V_{+}(I)\subset V_{+}(J)\Leftrightarrow J\cap A_{+} \subset \sqrt{I}
      \end{equation*}
    \item[(3)] $\proj{A}=\varnothing \Leftrightarrow A_{+}$が冪零
  \end{itemize}
}{
  (1)$I$を素イデアルとする.$a,b\in A$が$ab\in I^{h}$で$a,b\notin I^{h}$を満たすとする.
  それぞれの斉次元への分解を
  \begin{equation*}
    a = \sum_{i = 0}^{n} a_{i},\quad b = \sum_{j = 0}^{m}b_{j},\quad a_{d},b_{d} \in A_{d}
  \end{equation*}
  とすると,$a_{n},b_{m} \notin I^{h}$となるように$a,b$を取り直す事ができる.実際$0\leq k \leq n$を$a_{k} \notin I^h$となる最大のものとすると
  \begin{equation*}
    (a-a_{n})b = ab - a_{n}b \in I^{h} 
  \end{equation*}
  なので,繰り返すと,
  \begin{equation*}
    (a - a_{n} - a_{n-1} - \cdots - a_{k+1})b \in I^{h}
  \end{equation*}
  で,$b_{m}$についても同様で$0\leq l \leq m$を$b_{l} \notin I^h$となる最大のものとする.このとき,
  \begin{equation*}
    a_{0} = a - a_{n} - a_{n-1} - \cdots - a_{k+1},\quad b_{0} = b - b_{m} - b_{m-1} - \cdots - b_{l+1}
  \end{equation*}
  とすると,$a_{0}b_{0}\in I^{h}$で$a_{0},b_{0} \in I^{h}$である.ここで
  \begin{equation*}
    a_{0}b_{0} = a_{k}b_{l} + (k+l\text{次未満の項})
  \end{equation*}
  となり,$a_{0}b_{0}\in I^{h}$より$a_{k}b_{l} \in I^{h} \subset I$だから$I$が素イデアルということより$a_{k},b_{l}\in I$よって$a_{k},b_{l} \in I^{h}$となる.
  よって矛盾する.
  \\
  (2)まず$(\Leftarrow)$を示す.つまり$J\cap A_{+}\subset \sqrt{I}$とする.
  任意の$\mathfrak{p}\in V_{+}(I)$に対して
  \begin{equation*}
    \mathfrak{p}\supset J\cap A_{+} \supset JA_{+}
  \end{equation*}

}

斉次元$f\in A$に対して
\begin{equation*}
  D_{+}(f) = \proj{A}\mysetminus V_{+}(fA)
\end{equation*}
これを\index{きほんかいしゅうごう@基本開集合}\index{principal open subset}
\textbf{基本開集合(principal open subset)}
という.基本開集合の族$\{D_{+}(f)\}_{f}$は$\proj{A}$の開基になっている.
\\
\Lemma{
  3.36
}{}

\Proposition{
  3.38
}{}

\Lemma{
  3.40
}{}

\Lemma{
  3.41
}{}

\Definition{
  3.42
}{}

\Lemma{
  3.43
}{}

\Corollary{
  3.44
}{}

