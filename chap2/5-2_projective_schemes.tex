
\subsection{Projective schemes}
\sectionmark{Projective schemes}
まず初めに次数環
\begin{equation*}
  A = \bigoplus_{n\in \mathbf{N}}A_{n}
\end{equation*}
を固定する.ここでイデアル$I \subset A$が斉次イデアルとは
\begin{equation*}
  I = \bigoplus_{n\in \mathbf{N}}(I\cap A_{n})
\end{equation*}
のときをいう.ここで
\begin{equation*}
  A/I = \left. \bigoplus_{n\in \mathbf{N}} A_{n}\right / \bigoplus_{n\in \mathbf{N}}(I\cap A_{n})
\end{equation*}
だが
$$
\begin{array}{rccc}
  \varphi \colon &\bigoplus_{n\in \mathbf{N}}A_{n}/(I\cap A_{n})                     &\longrightarrow& A/I                    \\
          & \rotatebox{90}{$\in$}&               & \rotatebox{90}{$\in$} \\
          & (x_{i} + I \cap A_{i})_{i}                    & \longmapsto   & (x_{i})_{i} + I
\end{array}
$$
とするとこれは全準同型で単射性は$(x_{i})_{i} + I = (y_{i})_{i} + I$とすると$(x_{i})_{i} - (y_{i})_{i}  = (x_{i} - y_{i})_{i}\in I$
と$x_{i} \in A_{i}$より$x_{i} - y_{i} \in I \cap A_{i}$でこれは単射であることを意味する.
よって,
\begin{equation*}
  A/I = \bigoplus_{n \in \mathbf{N}}A_{n}/(I \cap A_{n})
\end{equation*}
である.ここで$\proj{A}$を次のように定義しよう.
\begin{equation*}
  \proj{A}:= \left\{\mathfrak{p} \in \spec{A}\mid \mathfrak{p}は斉次イデアルでA_{+} \not \subset \mathfrak{p} \right\}
\end{equation*}
とおく.
%($A_{+}\not \subset \mathfrak{p}\Leftrightarrow A_{+}\cap \mathfrak{p}\not \subset A_{+}$であることに注意)
ただし
\begin{equation*}
  A_{+} := \bigoplus_{n > 0}A_{n}
\end{equation*}
である.あとで$\proj{A}$にスキームの構造が入ることを示そう.\\
任意の斉次イデアル$I\subset A$に対して
\begin{equation*}
  V_{+}(I) := \{\mathfrak{p} \in \proj{A}\mid I \subset \mathfrak{p}\}
\end{equation*}
と定義する.このとき
\begin{align}
  \bigcap_{\mu}V_{+}(I_{\mu}) &= V_{+}(\sum_{\mu}I_{\mu})\\
  V_{+}(I)\cup V_{+}(J) &= V_{+}(I\cap J)\\
  V_{+}(A) &= \varnothing\\
  V_{+}(0) &= \proj{A}
\end{align}
が成り立つ.実際(1.1)から示そう.
\begin{align*}
  I_{\lambda} \subset \sum_{\mu}I_{\mu}
\end{align*}
なので
\begin{equation*}
  V_{+}(\sum_{\mu}I_{\mu}) \subset V_{+}(I_{\lambda})
\end{equation*}
である.よって
\begin{equation*}
  V_{+}(\sum_{\mu}I_{\mu}) \subset \bigcap_{\mu}V_{+}(I_{\mu})
\end{equation*}
逆に$\mathfrak{p} \in \bigcap V_{+}(I_{\mu})$とすると任意の$\mu$に対して$I_{\mu} \subset \mathfrak{p}$なので
$\sum I_{\mu}\subset \mathfrak{p}$ が成り立ち逆の包含関係もわかる.\\
(1.2)は$\mathfrak{p} \in V_{+}(I)$なら$I \subset \mathfrak{p}$なので$I\cap J \subset \mathfrak{p}$
だから$V_{+}(I)\subset V_{+}(I\cap J)$で同様に$V_{+}(J) \subset V_{+}(I\cap J)$なので
\begin{equation*}
  V_{+}(I)\cup V_{+}(J) \subset V_{+}(I\cap J)
\end{equation*}
逆に$\mathfrak{p} \in V_{+}(I\cap J)$なら$I\cap J \subset \mathfrak{p}$で$I\not\subset \mathfrak{p}$なら$a\in I$かつ$a\notin \mathfrak{p}$なる元がある.しかし,任意の$b\in J$に対して
$ab \in I\cap J$なので$ab \in \mathfrak{p}$で今$a\notin \mathfrak{p}$なので$b\in \mathfrak{p}$
である.よって$J\subset \mathfrak{p}$なので$V_{+}(I\cap J)\subset V_{+}(J)$だから逆の包含関係もわかる.
残り二つは自明である.\\
$\spec$の場合と同様に$\proj{A}$にも$\{V_{+}(I)\}_{I}$を閉集合族とする位相を入れることにする.
この位相を同様にZariski位相ということにする.\\
$I$を$A$の任意のイデアルとすると,$I$に伴う斉次イデアル$I^{h} = \bigoplus(I\cap A_{n})$(ここで$h$乗ではなく単なる記号であることに注意)が定義できる.
\Lemma{
  $I,J$を次数環$A$のイデアルとする.このとき以下が成り立つ.
  \begin{itemize}
    \item[(1)] $I$が素イデアルならそれに伴う斉次イデアル$I^{h}$も素イデアル.
    \item[(2)] $I$と$J$が斉次イデアルとする.このとき
      \begin{equation*}
        V_{+}(I)\subset V_{+}(J)\Leftrightarrow J\cap A_{+} \subset \sqrt{I}
      \end{equation*}
    \item[(3)] $\proj{A}=\varnothing \Leftrightarrow A_{+}$が冪零
  \end{itemize}
}{
  (1)$I$を素イデアルとする.$a,b\in A$が$ab\in I^{h}$で$a,b\notin I^{h}$を満たすとする.
  それぞれの斉次元への分解を
  \begin{equation*}
    a = \sum_{i = 0}^{n} a_{i},\quad b = \sum_{j = 0}^{m}b_{j},\quad a_{d},b_{d} \in A_{d}
  \end{equation*}
  とすると,$a_{n},b_{m} \notin I^{h}$となるように$a,b$を取り直す事ができる.実際$0\leq k \leq n$を$a_{k} \notin I^h$となる最大のものとすると
  \begin{equation*}
    (a-a_{n})b = ab - a_{n}b \in I^{h} 
  \end{equation*}
  なので,繰り返すと,
  \begin{equation*}
    (a - a_{n} - a_{n-1} - \cdots - a_{k+1})b \in I^{h}
  \end{equation*}
  で,$b_{m}$についても同様で$0\leq l \leq m$を$b_{l} \notin I^h$となる最大のものとする.このとき,
  \begin{equation*}
    a_{0} = a - a_{n} - a_{n-1} - \cdots - a_{k+1},\quad b_{0} = b - b_{m} - b_{m-1} - \cdots - b_{l+1}
  \end{equation*}
  とすると,$a_{0}b_{0}\in I^{h}$で$a_{0},b_{0} \in I^{h}$である.ここで
  \begin{equation*}
    a_{0}b_{0} = a_{k}b_{l} + (k+l\text{次未満の項})
  \end{equation*}
  となり,$a_{0}b_{0}\in I^{h}$より$a_{k}b_{l} \in I^{h} \subset I$だから$I$が素イデアルということより$a_{k},b_{l}\in I$よって$a_{k},b_{l} \in I^{h}$となる.
  よって矛盾する.\\
  (2)まず$(\Leftarrow)$を示す.つまり$J\cap A_{+}\subset \sqrt{I}$とする.
  任意の$\mathfrak{p}\in V_{+}(I)$に対して
  \begin{equation*}
    \mathfrak{p}\supset J\cap A_{+} \supset JA_{+}
  \end{equation*}
  ここで,$\mathfrak{p}$は素イデアルで,$A_{+}\not \subset \mathfrak{p}$だから$J\subset \mathfrak{p}$
  となる.よって$\mathfrak{p} \in V_{+}(J)$が成り立つ.\\
  次に$(\Rightarrow)$を示す.$V_{+}(I)\subset V_{+}(J)$とする.任意の$\mathfrak{p} \in V(I)$に対して,これに伴う斉次イデアル$\mathfrak{p}^{h}$は素イデアルで$I$を含む.
  もし,$A_{+}\not \subset \mathfrak{p}^{h}$なら,$\mathfrak{p}^{h} \in V_{+}(I)$.
  したがって,$\mathfrak{p} \supset \mathfrak{p}^{h} \supset J\cap A_{+}$.$A_{+}\subset \mathfrak{p}^{h}$であっても$\mathfrak{p}\supset \mathfrak{p}^{h} \supset J\cap A_{+}$となる.従って,
  \begin{equation*}
    J\cap A_{+} \subset \bigcap_{\mathfrak{p} \in V(I)}\mathfrak{p} = \sqrt{I}
  \end{equation*}
  となる.\\
  (3)$\proj{A} = \varnothing$は$V_{+}(0) \subset V_{+}(A_{+})$と同値.(2)より,これは更に,$A_{+}\subset \sqrt{0}$と同値である.これは$A_{+}$が冪零ということである.
}

斉次元$f\in A$に対して
\begin{equation*}
  D_{+}(f) = \proj{A}\mysetminus V_{+}(fA)
\end{equation*}
これを\index{きほんかいしゅうごう@基本開集合}\index{principal open subset}
\textbf{基本開集合(principal open subset)}
という.基本開集合の族$\{D_{+}(f)\}_{f}$は$\proj{A}$の開基になっている.実際,斉次イデアルは斉次元の集合で生成されるので,ある斉次元$f_{i}$があって,
\begin{equation*}
  \varnothing = V_{+}(A_{+}) = V_{+}(\sum_{i}f_{i}) = \bigcap_{i}V_{+}(f_{i})
\end{equation*}
となる.よって$\proj{A} = \bigcup_{i}D_{+}(f_{i})$となる.また,任意の斉次元$g\in A$に対して,$D_{+}(g) = \bigcup_{i}D_{+}(gf_{i})$で,$gf_{i} \in A_{+}$となる.\\
斉次元$f\in A$に対して,
\begin{equation*}
  A_{(f)} = \left\{ \frac{a}{f^{N}}\ \middle\vert \ a\in A,\ N\geq 0,\ \deg{a} = N\deg{f} \right\}
\end{equation*}
とする.$A_{(f)}$の元を\textbf{$A_{f}$の次数$0$の元(elements of degree $0$ of $A_{f}$)}という.これはなぜかというと,
\begin{equation*}
  A_{f} = \left\{\frac{a}{f^{n}}\ \middle\vert \ a\in A,n\in \mathbb{Z}\right\}
        = \bigoplus_{n \in \mathbb{Z}}\left\{ \frac{a}{f^{k}}\ \middle\vert \ a\in A,\deg{a} - k\deg{f} = n \right\}
\end{equation*}
であるから($A$は次数環であることに注意)この次数$0$の部分は
\begin{equation*}
  \left\{ \frac{a}{f^{k}} \ \middle\vert \ a\in A,\deg{a} - k\deg{f} = 0\right\} = A_{(f)}
\end{equation*}
ということである.よって,$A_{f}$は次数環であり,次数$A_{(f)}$代数である.\\
例えば,$A = k[T_{0},\cdots,T_{n}]$としたとき,$A_{(T_{i})} = k[T_{i}^{-1}T_{j}]_{0\leq j\leq n}$である.
\Lemma{
  $f\in A_{+}$を次数$r$の斉次元とする.
  \begin{itemize}
    \item[(1)] 標準的な同相写像$\theta:D_{+}(f) \to \spec{A_{(f)}}$がある.
    \item[(2)] $D_{+}(g) \subset D_{+}(f)$で$\alpha = g^{r}f^{-\deg{g}}\in A_{(f)}$を満たすならば,$\theta(D_{+}(g)) = D(\alpha)$. 
    \item[(3)] 同型$(A_{(f)})_{\alpha} \simeq A_{(g)}$から引き起こされる標準的な射$A_{(f)} \to A_{(g)}$がある.
    \item[(4)] $I$を$A$の斉次イデアルとする.このとき$\theta(V_{+}(I)\cap D_{+}(f))=V(I_{(f)})$なる閉集合である.ただし,$I_{(f)} = IA_{f}\cap A_{(f)}$である.
    \item[(5)] $\{h_{1},\cdots,h_{n}\}$を$I$を生成する斉次元の集合とする.
    このとき,任意の$f\in B_{1}$に対して,$I_{(f)}$は$h_{i}/f^{\deg{h_{i}}}$で生成される.  
  \end{itemize}
}{
  (1)$\proj{A}$は$\spec{A}$の部分集合である.更に,任意の$f\in A$に対して,その斉次元への分解を$f = f_{0} + f_{1} + \cdots + f_{d}\ (f_{i} \in A_{i})$とすると,
  \begin{equation*}
    V(f) \cap \proj{A} = V_{+}(f) = \bigcap_{i}V_{+}(f_{i})
  \end{equation*}
  となるから,
  \begin{equation*}
    D(f) \cap \proj{A} = \bigcup_{i}D_{+}(f_{i})
  \end{equation*}
  となる.よって,$\proj{A}$の位相は$\spec{A}$の相対位相と一致する.$\theta:D_{+}(f)\to \spec{A_{(f)}}$
  を,標準的な射$D(f) = \spec{A_{f}} \to \spec{A_{(f)}}$の$D_{+}(f)$への制限とする.これは連続である.\\
  次に,$\theta$が全射であることを示そう.$A_{f}$は次数$A_{(f)}$代数であること,また,次数$n\ (n\in \mathbf{Z})$の斉次元は$af^{-N}\ (a\in A, \deg{a} = Nr+n)$の形になることに注意しよう.(この命題の前に述べたことである.)
  $\mathfrak{q}\in \spec{A_{(f)}}$を取る.
  まず容易に$\sqrt{\mathfrak{q}A_{f}}$が$A_{f}$の素イデアルであることがわかる.
  実際,$a,b\in A_{f}$で,$ab\in \sqrt{\mathfrak{q}A_{f}}$とすると,
  定義から$(ab)^{k}\in \mathfrak{q}A_{f}$となる$k\geq 1$がある.
  ここで,$a^{r}f^{-\deg{a}}$は次数$0$の元なので,
  \begin{equation*}
    (a^{r}f^{-\deg{a}})^{k}(b^{r}f^{-\deg{b}})^{k}\in \mathfrak{q}
  \end{equation*}
  である.いま,$\mathfrak{q}$は素イデアルなので,
  \begin{equation*}
    (a^{r}f^{-\deg{a}})^{k} \in \mathfrak{q}\ \text{or}\ (b^{r}f^{-\deg{b}})^{k} \in \mathfrak{q}
  \end{equation*}
  である.もし前者がなりたつとすると,$a^{r}f^{-\deg{a}}\in \mathfrak{q}$である.\\
  よって,$(a^{r}f^{-\deg{a}})(f^{\deg{a}}) = a^{r} \in \mathfrak{q}A_{f}$である.
  したがって,$a\in \sqrt{\mathfrak{q}A_{f}}$となる.
  また,$\mathfrak{q}A_{f}$は斉次イデアルだから,その根基$\sqrt{\mathfrak{q}A_{f}}$も斉次イデアルである.
  $\rho:A\to A_{f}$を標準的な射とする.これは次数付き環の射
  \footnote{次数付き環の射とは$\deg$とその射が整合的つまり,斉次元$a$に対して$\deg{f(a)} = \deg{a}$となることであろう.}
  である.
  ここで,$\mathfrak{p} := \rho^{-1}(\sqrt{\mathfrak{q}A_{f}})$について考えると,これは
  容易に$A$の斉次素イデアル
  \footnote{斉次イデアルの次数付き環の射による逆像は斉次イデアルである.実際,$a \in f^{-1}(\mathfrak{p})$で,$a=\sum_{k}a_{k}$とすると,$f(a) = \sum_{k}f(a_{k})\in \mathfrak{p}$.$\mathfrak{p}$の斉次性より,$f(a_{k})\in \mathfrak{p}_{k}$.よって,$a_{k} \in f^{-1}(\mathfrak{p}_{k}) \subset f^{-1}(\mathfrak{p})$である.}
  で,$\mathfrak{p}\in D_{+}(f)$となることがわかる.
  \footnote{$\rho$は次数付き環の射なので,次数$0$の部分の引き戻しは次数$0$の部分に入る.よって,$A_{+}\not \subset \mathfrak{p}$}
  
}

\Proposition{
  3.38
}{}

\Lemma{
  3.40
}{}

\Lemma{
  3.41
}{}

\Definition{
  3.42
}{}

\Lemma{
  3.43
}{}

\Corollary{
  3.44
}{}

