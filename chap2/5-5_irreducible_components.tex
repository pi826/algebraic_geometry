
\subsection{Irreducible Components}
\sectionmark{Irreducible Components}

\Definition{
  位相空間$X$が\index{きやく@既約}\index{irreducible}
  \textbf{既約(irreducible)}とは,$X$の閉集合$X_{1},X_{2}$を用いて
  \begin{equation*}
    X = X_{1}\cup X_{2}
  \end{equation*}
  となるとき,$X_{1} = X$または$X_{2} = X$が成り立つときを言う.
}

\Lemma{
  位相空間$X$には包含関係で極大な既約部分空間が存在する.
}{
  既約部分空間の集合$\mathscr{S} = \{U\subset X\mid U\text{:irreducible}\}$
  をとると,$\varnothing \in \mathscr{S}$なので空ではない.また,包含関係で半順序である.
  全順序部分集合$\{V_{\lambda}\}_{\lambda\in \Lambda}\subset \mathscr{S}$に対して
  \begin{equation*}
    U_{\infty} = \bigcup_{\lambda \in \Lambda}V_{\lambda}
  \end{equation*}
  と置くと,これは既約である.したがって,$U_{\infty}\in \mathscr{S}$であるからZornの補題より
  極大な既約部分空間が存在する.
}


\Definition{
  位相空間$X$の極大な既約部分空間を$X$の\index{きやくせいぶん@既約成分}\index{irreducible components}
  \textbf{既約成分(irreducible components)}という.
}

\Remark{
  位相空間$X$の既約成分全体の和集合は$X$に等しい.なぜなら,一点集合$\{x\}\subset X$は既約
  なので,これを含む既約成分が存在するからである.
}{}

\Lemma{
  位相空間$X$の部分集合$U$に対して以下は同値
  \begin{itemize}
    \item[(1)] $U$は既約
    \item[(2)] $\overline{U}$は既約 
  \end{itemize}
}{
  $U$の閉集合$U_{1},U_{2}$を用いて
  \begin{equation*}
    U = U_{1} \cup U_{2} 
  \end{equation*}
  とかけたとき,
  \begin{equation*}
    \overline{U} = \overline{U_{1}\cup U_{2}} = \overline{U_{1}} \cup \overline{U_{2}} = U_{1}\cup U_{2} = U
  \end{equation*}
  なのでよい.
}

\Corollary{
  既約成分は閉である.
}{}

\Lemma{
  位相空間$X$に対して以下は同値
  \begin{itemize}
    \item[(1)] $X$は既約
    \item[(2)] 任意の空でない開部分集合$U_{1},U_{2}$に対して$U_{1}\cap U_{2}$は空でない.
  \end{itemize}
}{
  $(\Rightarrow)$
  $X$が既約とし,$U_{1},U_{2}$を空でない開部分集合をとる.今$U_{1}\cap U_{2}$が空であると仮定すると,
  \begin{align*}
    X 
    &= X\mysetminus (U_{1}\cap U_{2})\\
    &= (X\mysetminus U_{1}) \cup (X\mysetminus U_{2}) 
  \end{align*}
  $X\mysetminus U_{i}\ (i = 1,2)$は閉なので既約性より$X\mysetminus U_{1} = X$または$X\mysetminus U_{2} = X$が成り立つが,これは$U_{i}$が空でないことに矛盾する.\\
  $(\Leftarrow)$$X_{1},X_{2}$を$X$の閉部分集合とし,$X=X_{1}\cup X_{2}$とする.今$X_{1} \neq X$かつ$X_{2}\neq X$と仮定すると,
  $X\mysetminus X_{i}\neq \varnothing\ (i = 1,2)$でありそれぞれ開なので,前提より
  \begin{equation*}
    (X\mysetminus X_{1})\cap (X\mysetminus X_{2}) = X\mysetminus (X_{1}\cup X_{2}) \neq \varnothing
  \end{equation*}
  したがって,$X_{1}\cup X_{2} \neq X$であり矛盾する.
}

\Proposition{
  $X$を位相空間とする.
  \begin{itemize}
    \item[(1)] $X$が既約ならば$X$の空でない開集合$U$は稠密かつ既約である.
    \item[(2)] $\{X_{i}\}_{i}$を$X$の既約成分,$U$を$X$の開集合とする.このとき$\{X_{i}\cap U\}_{i}$は$U$の既約成分である. 
    \item[(3)] $X$は有限個の既約な閉部分集合$Z_{j}$に分解されるとする.このとき任意の$X$の既約成分$Z$はある$Z_{j}$に等しい.更に,もし任意の$i,j$に対して$Z_{i}\not\subset Z_{j}$ならば,
    $\{Z_{j}\}_{j}$が$X$の既約成分の全体である.
  \end{itemize}
}{
  $(1)\ $$U$を空でない開集合とし,$\overline{U}$を$U$の閉包とする.このとき
  \begin{equation*}
    X = (X\mysetminus U) \cup U = (X\mysetminus U) \cup \overline{U} 
  \end{equation*}
  $X\mysetminus U \subsetneq X$なので既約性より$X = \overline{U}$を得る.よって稠密である.
  最後に既約であることは$U$の空でない開部分集合$U_{1},U_{2}$に対してそれぞれ$X$で稠密であるので任意の開集合との交わりを持つ.したがって$U_{1}\cap U_{2}$は空でない.
  したがって,先程の補題より$U$は既約である.\\
  $(2)\ $$U$は空でないとしてよい.$X_{i}$を$X_{i}\cap U\neq \varnothing$なる$X$の既約成分とする.もし,$(X_{i}\cap U)\subset Z$なる既約な閉集合$Z\subset U$があるとすると,
  $X_{i}\cap U$は既約部分空間$X_{i}$の空でない開集合なので$(1)$より稠密なので
  \begin{equation*}
    X_{i} = \overline{X_{i}\cap U}
  \end{equation*}
  包含関係より
  \begin{equation*}
    X_{i} = \overline{X_{i}\cap U}\subset \overline{Z} = Z
  \end{equation*}
  $Z$は既約なので$X_{i}$の極大性から$Z=X_{i}$.よって,
  \begin{equation*}
    Z\cap U = Z = X_{i}\cap U
  \end{equation*}
  従って,$X_{i}\cap U$は$U$の既約成分である.逆に,$Z$を$U$の既約成分とすると,
  ある$X$の既約成分$X_{i}$があって,$Z\subset X_{i}$を満たす.従って,
  \begin{equation*}
    Z\subset X_{i}\cap U
  \end{equation*}
  よって,$Z=X_{i}\cap U$であるからよい.\\
  $(3)\ $いま
  \begin{equation*}
    Z = \bigcup_{j} Z_{j}\cap Z
  \end{equation*}
  である.$Z_{j}$の数で帰納法を行う.一つのときはよい.
}

\Definition{
  スキームが\index{きやく@既約}\index{irreducible}\textbf{既約(irreducible)}とは,
  位相空間として既約なことをいう.
}

\Proposition{
  $X = \spec{A}$とし,$I$を$A$のイデアルとする.
  \begin{itemize}
    \item[(1)] $V(I)$が既約$\Longleftrightarrow$$\sqrt{I}$が素イデアル.
    \item[(2)] $\{\mathfrak{p}_{i}\}_{i}$を$A$の極小素イデアルとする.このとき,$\{V(\mathfrak{p}_{i})\}_{i}$は$X$の既約成分である.
    \item[(3)] $X$が既約$\Longleftrightarrow$$A$は唯一の極小素イデアルを持つ.特に,$A$が整域ならば,$X$は既約.
  \end{itemize}
}{
  $(1)\ $まず$\sqrt{I}$が素イデアルだとする.いま$V(I) = V(J_{1}) \cup V(J_{2})$とすると,
  $\sqrt{I} = \sqrt{J_{1}J_{2}} \supset J_{1}J_{2}$.仮定から$J_{1}\subset \sqrt{I}$または$J_{2}\subset \sqrt{I}$が成り立つ.
  よって,$V(J_{1})\supset V(I)$または$V(J_{2})\supset V(I)$が成り立つ.
  よって,$V(J_{1}) = V(I)$または$V(J_{2}) = V(I)$が成り立つ.従って,$V(I)$は既約.
  逆に,$\sqrt{I}$が素イデアルでないとすると,ある$a,b\in A\mysetminus \sqrt{I}$があって,
  $ab\in \sqrt{I}$.よって,$V(I) = (V(a)\cap V(I))\cup (V(b) \cap V(I))$でそれぞれ$V(I)$
  と等しくない.従って,$V(I)$は既約でない.$(2),(3)$は$(1)$から容易にわかる.
}

\Proposition{
  ネータースキーム$X$は唯一の有限個の既約成分を持つ.
}{
  
}

