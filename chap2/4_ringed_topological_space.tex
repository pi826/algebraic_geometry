
\Section{Ringed Topological Space}
\sectionmark{Ringed Topological Space}

\Definition{
  \index{きょくしょかんつきくうかん@局所環付き空間}\index{locally ringed space}
  \textbf{局所環付き空間}とは位相空間$X$と$X$上の環の層$\mathcal{O}_{X}$の組$(X,\mathcal{O}_{X})$
  で、任意の$x\in X$に対して$\mathcal{O}_{X,x}$が局所環となるものをいう。また、この$\mathcal{O}_{X}$
  を$(X,\mathcal{O}_{X})$の
  \index{こうぞうそう@構造層}\index{structure sheaf}
  \textbf{構造層(structure sheaf)}という。また$(X,\mathcal{O}_{X})$を単に
  $\mathcal{O}_{X}$と書くことがある。\\
  また、$\mathcal{O}_{X,x}$の唯一の極大イデアル$\mathfrak{m}_{x}$に対して
  その剰余体$\mathcal{O}_{X,x}/\mathfrak{m}_{x}$を
  \index{てんでのじょうよたい@点での剰余体}\index{residue field at point}
  \textbf{$X$の点$x$での剰余体(residue field of $X$ at $x$)}といって$k(x)$と書く。
}

\Definition{
\index{きょくしょかんつきくうかん@局所環付き空間!のしゃ@の射}\index{locally ringed space!morphism}
局所環付き空間の射とは
\begin{equation*}
  (f,f^{\#}): (X,\mathcal{O}_{X}) \to (Y,\mathcal{O}_{Y})
\end{equation*}
とは連続写像$f:X \to Y$と環の層の射$f^{\#}:\mathcal{O}_{Y} \to f_{*}\mathcal{O}_{X}$の組$(f,f^{\#})$で、
任意の$x\in X$に対して$f^{\#}_{x} : \mathcal{O}_{Y,f(x)} \to \mathcal{O}_{X,x}$は局所射となるものをいう。(つまり$f^{\#}_{x}(\mathfrak{m}_{Y,f(x)})\subset \mathfrak{m}_{X,x}$を満たす.)
}
Prop:\ref{Prop:1.3.13}より
\begin{equation*}
  f^{\#} : \mathcal{O}_{Y} \to f_{*}\mathcal{O}_{X}
\end{equation*}
を考えることは
\begin{equation*}
  f^{\#} : f^{-1}\mathcal{O}_Y \to \mathcal{O}_{X}
\end{equation*}
を考えることに等しい.Def:\ref{Def:1.4.2}の$f^{\#}_{x}$は下の式で考えている.

\Definition{
\index{きょくしょかんつきくうかん@局所環付き空間!かいはめこみ@開はめ込み}\index{locally ringed space!open immersion}
射$(f,f^{\#}):(X,\mathcal{O}_{X}) \to (Y,\mathcal{O}_{Y})$が\textbf{開はめ込み(open immersion)}(resp. 
\index{きょくしょかんつきくうかん@局所環付き空間!へいはめこみ@閉はめ込み}\index{locally ringed space!closed immersion}
\textbf{閉はめ込み(closed immersion)})
とは連続写像$f$が開はめ込み(resp. 閉はめ込み)
\footnote{$f:X\to Y$が(位相的)開(閉)はめ込みとは$X$と$f(X)$が同相で$f(X)$が開(閉)集合のときをいう。}
でかつ任意の$x\in X$に対して$f^{\#}_{x}$が同型(resp. 全射)のときをいう。
}

\Definition{
  \index{いであるそう@イデアル層}\index{sheaf of ideals}
  $(X,\mathcal{O}_{X})$を局所環付き空間とする。$\mathcal{J}$が$\mathcal{O}_{X}$の
  \textbf{イデアル層(sheaf of ideals of $\mathcal{O}_{X}$)}とは任意の開集合$U$に対して
  $\mathcal{J}(U)$が$\mathcal{O}_{X}(U)$のイデアルになっているときをいう。
}

\Lemma{
$(X,\mathcal{O}_{X})$を局所環付き空間とする。$\mathcal{J}$を$\mathcal{O}_{X}$のイデアル層とする。
そして、
\begin{equation*}
  V(\mathcal{J}) = \{x \in X\ |\ \mathcal{J}_{x} \neq \mathcal{O}_{X,x}\}
\end{equation*}
とおく。(ちなみに上の諸々から$\mathcal{J}_{x} \subset \mathcal{O}_{X,x}$が分かる。)\\
$j:V(\mathcal{J}) \hookrightarrow X$を包含写像とする。すると
\begin{itemize}
  \item[---] $V(\mathcal{J})$は$X$の閉集合
  \item[---] $(V(\mathcal{J}),j^{-1}(\mathcal{O}_{X}/\mathcal{J}))$は局所環付き空間
  \item[---] $j^{\#}$は自然な全射
        \begin{equation*}
          \mathcal{O}_{X} \longrightarrow \mathcal{O}_{X}/\mathcal{J} = j_{*}(j^{-1}(\mathcal{O}_{X}/\mathcal{J}))
        \end{equation*}
        で
        $(j,j^{\#}):(V(\mathcal{J}),j^{-1}(\mathcal{O}_{X}/\mathcal{J})) \to (X,\mathcal{O}_{X})$は閉はめ込みである。
\end{itemize}
}{
\Claim{1}$V(\mathcal{J})$は$X$の閉集合\\
$x\in X\mysetminus V(\mathcal{J}) = \{x\in X\ |\ \mathcal{J}_{x} = \mathcal{O}_{X,x}\}$
に対して$f_{x} = 1$なる$x$の開近傍$U$と$f\in \mathcal{J}(U)$をとる。
つまり$f|_{V} = 1|_{V} = 1$なる$x$の開近傍$V \subset U$がある。
すると任意の$y \in V$に対して$f_{y} = 1 \in \mathcal{J}_{y}$となって,この$y$に対して
$\mathcal{J}_{y} = \mathcal{O}_{X,y}$なので
$V\subset X \mysetminus V(\mathcal{J})$となって$X\mysetminus V(\mathcal{J})$が開であることがわかる。\\
\Claim{2}$(V(\mathcal{J}),j^{-1}(\mathcal{O}_{X}/\mathcal{J}))$は局所環付き空間\\
任意の$x\in V(\mathcal{J})$に対して
\begin{equation*}
  (j^{-1}(\mathcal{O}_{X}/\mathcal{J}))_{x} = (\mathcal{O}_{X}/\mathcal{J})_{x} = \mathcal{O}_{X,x}/\mathcal{J}_{x}
\end{equation*}
は局所環。残りは自明。
}

\Proposition{
  $f:X \to Y$を局所環付き空間の閉はめ込みとする。$Z$を局所環付き空間$V(\mathcal{J})$とする。
  ただし、$\mathcal{J} = \ker f^{\#}\subset \mathcal{O}_{Y}$.すると$X\simeq Z$を自然な閉はめ込み
  $Z \hookrightarrow Y$から得る。
}{
  まず次の完全列
  \begin{equation*}
    0 \longrightarrow \mathcal{J} \longrightarrow \mathcal{O}_{Y} \longrightarrow f_{*}\mathcal{O}_{X} \longrightarrow 0
  \end{equation*}
  からProp:\ref{Prop:1.3.11}より任意の$y \in Y$に対して
  \begin{equation*}
    \mathcal{O}_{Y,y}/\mathcal{J}_{y} = (f_{*}\mathcal{O}_{X})_{y}
  \end{equation*}
  を得る。よって
  \begin{equation*}
    (f_{*}\mathcal{O}_{X})_{y} = \left\{
    \begin{alignedat}{2}
        & \quad 0 \qquad                           & y & \in Y \mysetminus V(\mathcal{J}) \\
        & \mathcal{O}_{Y,y}/\mathcal{J}_{y} \qquad & y & \in V(\mathcal{J})
    \end{alignedat}
    \right.\qquad \cdots (*)
  \end{equation*}
  を得る。$f(X)$は$Y$の閉集合なので$x\in Y\mysetminus f(X)$の開近傍$U$で
  \begin{equation*}
    f(X)\cap U = \varnothing
  \end{equation*}
  となるものがとれる。よって
  \begin{equation*}
    f_{*}\mathcal{O}_{X}(U) = \mathcal{O}_{X}(f^{-1}(U)) = \mathcal{O}_{X}(\varnothing) = 0
  \end{equation*}
  したがって、
  \begin{equation*}
    (f_{*}\mathcal{O}_{X})_{x} = 0
  \end{equation*}
  また、$y\in f(Y)$の開近傍$U$に対して$f$での引き戻し$f^{-1}(U)$は$y=f(x)$なる$x\in X$の開近傍である。
  これを$V$とおく。逆に、$f$は閉はめ込みなので、$X$は$f(X)$と同相なので$X$に自然に$Y$の相対位相が入る。
  つまり、任意の$x\in X$の開近傍$U$に対して$y=f(x)\in Y$の開近傍$V$が存在して$f^{-1}(V)$とかける。
  よって、
  \begin{equation*}
    (f_{*}\mathcal{O}_{X})_{y} = \varinjlim_{U \ni y}f_{*}\mathcal{O}_{X}(U) = \varinjlim_{U \ni y}\mathcal{O}_{X}(f^{-1}(U)) = \varinjlim_{V \ni x}\mathcal{O}_{X}(V) = \mathcal{O}_{X,x}
  \end{equation*}
  つまり、
  \begin{equation*}
    (f_{*}\mathcal{O}_{X})_{y} = \left\{
    \begin{alignedat}{2}
        & \quad 0 \qquad           & y & \in Y \mysetminus f(X) \\
        & \mathcal{O}_{X,x} \qquad & y & = f(x)
    \end{alignedat}
    \right.
  \end{equation*}
  $(*)$と比較すれば
  \begin{equation*}
    V(\mathcal{J}) = f(X)
  \end{equation*}
  が分かる。
  なので、$j:Z \hookrightarrow Y$を包含写像とすると、$f$から誘導される同相写像$g:X \to Z$に対して
  \begin{equation*}
    f = j \circ g
  \end{equation*}
  で、
  \begin{equation*}
    j_{*}\mathcal{O}_{Z} = \mathcal{O}_{Y}/\mathcal{J} \simeq f_{*}\mathcal{O}_{X}
  \end{equation*}
  がわかる。容易に
  \begin{equation*}
    f_{*}\mathcal{O}_{X} = j_{*}g_{*}\mathcal{O}_{X}
  \end{equation*}
  が分かるので
  \begin{equation*}
    \mathcal{O}_{Z} = (j^{-1}\circ j)_{*}\mathcal{O}_{Z} = (j^{-1})_{*}j_{*}\mathcal{O}_{Z}\simeq (j^{-1})_{*}j_{*}g_{*}\mathcal{O}_{X} = (j^{-1}\circ j)_{*} g_{*}\mathcal{O}_{X} = g_{*}\mathcal{O}_{X}
  \end{equation*}
  である。よって、$g$は局所環付き空間の同型射である。$f=j\circ g$が局所環付き空間の射であることを確認するのは読者に委ねる。
}