
\Section{Zariski Topology}
$\spec{A}$を幾何的な対象に昇華するために,位相を導入しよう.まず,環$A$のイデアル$I$に対して
\begin{align*}
  V(I) &= \{\mathfrak{p} \in \spec{A}\mid I\subset \mathfrak{p}\}\\
  D(I) &= \spec{A}\mysetminus V(I) = \{\mathfrak{p} \in \spec{A}\mid I\not\subset \mathfrak{p}\}
\end{align*}
更に,$f\in A$に対して
\begin{align*}
  V(f) &= \{\mathfrak{p} \in \spec{A}\mid Af \subset \mathfrak{p}\}\\
  D(f) &= \spec{A}\mysetminus V(Af) = \{\mathfrak{p} \in \spec{A}\mid Af \not\subset \mathfrak{p}\}
\end{align*}
と定義する.
また,$Af\subset \mathfrak{p}$より$af\in Af$は$af\in \mathfrak{p}$なので,$a=1$とすれば$f\in \mathfrak{p}$がわかり,
イデアルの定義より,
\begin{align*}
  V(f) &= \{\mathfrak{p} \in \spec{A} \mid f \in \mathfrak{p}\}\\
  D(f) &= \{\mathfrak{p} \in \spec{A} \mid f \notin \mathfrak{p}\}
\end{align*}
がわかる.次に$\{V(I)\}_{I}$を閉集合族とする位相が入ることを示そう.

\Proposition{
  $I,J,I_{\lambda}(\lambda \in \Lambda)$を環$A$のイデアルとする.このとき以下の等式が成り立つ.
  \begin{align}
    V(A) &= \varnothing\\
    V(0) &= \spec{A}\\
    V(I)\cup V(J) &= V(I \cap J)\\
    \bigcap_{\lambda \in \Lambda}V(I_{\lambda}) &= V(\sum_{\lambda \in \Lambda}I_{\lambda})
  \end{align}
}{
  等式(1.1),(1.2)は簡単にわかる.なぜなら$A$の任意の素イデアル$\mathfrak{p}$
  に対して$\mathfrak{p} \subsetneq A$
  なので$\mathfrak{p} \notin V(A)$なので(1.1)を得る.また,$0\in \mathfrak{p}$なので$\mathfrak{p} \in V(0)$なので(1.2)を得る.\\
  (1.3)については,$\mathfrak{p} \in V(I)$なら$I\subset \mathfrak{p}$
  なので,$I\cap J \subset \mathfrak{p}$を得る.よって,
  $V(I)\subset V(I\cap J)$で,同様に$V(J)\subset V(I \cap J)$である.
  よって,
  \begin{equation*}
    V(I)\cup V(J) \subset V(I\cap J)
  \end{equation*}
  である.逆に$\mathfrak{p} \in V(I\cap J)$なら$I\cap J \subset \mathfrak{p}$
  で,$I\not\subset \mathfrak{p}$なら$a\in I$かつ$a\notin \mathfrak{p}$
  なる元$a$がある.しかし,任意の$b\in J$に対して$ab \in I\cap J$なので,
  $ab\in \mathfrak{p}$で,いま$a\notin \mathfrak{p}$なので$b\in \mathfrak{p}$である.よって$J\subset \mathfrak{p}$
  なので$V(I \cap J)\subset V(J)$だから逆の包含関係も成り立つ.\\
  次に(1.4)は$\mu \in \Lambda$に対して
  \begin{equation*}
    I_{\mu} \subset \sum_{\lambda \in \Lambda}I_{\lambda}
  \end{equation*}
  なので
  \begin{equation*}
    V(\sum_{\lambda \in \Lambda}I_{\lambda}) \subset V(I_{\mu})
  \end{equation*}
  である.これが任意の$\mu \in \Lambda$で成り立つので
  \begin{equation*}
    V(\sum_{\lambda \in \Lambda}I_{\lambda}) \subset \bigcap_{\lambda \in \Lambda}V(I_{\lambda})
  \end{equation*}
  である.逆に$\mathfrak{p} \in \bigcap_{\lambda}V(I_{\lambda})$とすると
  任意の$\lambda \in \Lambda$に対して$I_{\lambda} \subset \mathfrak{p}$
  なので$\sum_{\lambda}I_{\lambda}\subset \mathfrak{p}$
  が成り立ち逆の包含関係もわかる.
}

\Corollary{
  有限個の環$A$のイデアル$\{I_{k}\}_{k = 1,\dots,n}$に対して
  \begin{equation*}
    \bigcup_{k = 1}^{n}V(I_{k}) = V(\bigcap_{k = 1}^{n}I_{k})
  \end{equation*}
  が成り立つ.
}{}

\Definition{
  $\{V(I)\}_{I}$を閉集合族とする$\spec{A}$の位相を
  \index{ざりすきいそう@Zariski位相}\index{Zariski topology}
  \textbf{Zariski位相(Zariski topology)}という.
}