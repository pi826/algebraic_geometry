
\subsection{Noetherian schemes, algebraic varieties}

\Definition{
  スキーム$X$の各点$x$のアフィン開近傍$\spec{A_{x}}$として,$A_{x}$がネーター環であるものがとれるとき
  \index{きょくしょねーたーすきーむ@局所ネータースキーム}\index{locally noetherian scheme}
  \textbf{局所ネータースキーム(locally noetherian scheme)}という.更に$X$が準コンパクトであれば
  \index{ねーたーすきーむ@ネータースキーム}\index{noetherian scheme}
  \textbf{ネータースキーム(noetherian scheme)}という.
}

\Proposition{
  $X$をネータースキームとする.
  \begin{itemize}
    \item[(1)] $X$の任意の開(閉)部分スキームはネーターである.
    \item[(2)] 任意の点$x\in X$に対して$\mathcal{O}_{X,x}$はネーター
    \item[(3)] 任意のアフィン開集合$U$に対して$\mathcal{O}_{X}(U)$はネーター  
  \end{itemize}
}{
  $X$はネーター的なので有限個のアフィン開集合$\{X_{i}\}$で被覆され$\mathcal{O}_{X}(X_{i})$はネーター環であるようにとる.\\
  (1)$Z$を$X$の開(閉)部分スキームとする.$Z\cap X_{i}$がネーター環であることを示せば十分である.
  また$Z\cap X_{i}$は$X_{i}$の開(閉)部分スキームであるので,結局$X$をアフィンスキームとしてよい.
  よって,$X=\spec{A}$とおく.もし$Z$が開だとすると,$Z=X\mysetminus V(I)$なる$I$がある.
  いま$A$はネーター環なので$I$は有限生成よって$Z$は有限個の基本開集合$D(f_{j})$で被覆され,
  更にその局所化(例えば$A_{f_{j}}$)はまたネーター的である.従って$Z$はネーターである.\\
  次に$Z$を閉のときはProp:\ref{Prop:1.5.12}よりよい.\\
  (2)環$\mathcal{O}_{X,x}$はネーター環の局所化なので再びネーター環になる.\\
  (3)上で見たように$U\cap X_{i}$は$X_{i}$の有限個のネーターアフィン開集合で被覆される.
  従って$U$は有限個のネーターアフィン開集合$U_{j}$で被覆されているとしてよい.
  $I$を$A=\mathcal{O}_{X}(U)$のイデアルとする.$I\mathcal{O}_{X}(U_{j})$
  は有限生成である.$J\mathcal{O}_{X}(U_{j}) = I\mathcal{O}_{X}(U_{j})$
  が任意の$j$で成り立つ有限生成なイデアル$J\subset I$が存在する.%感覚的にはわかるが,,
  任意の点$x\in U$に対して$J\mathcal{O}_{U,x} = I\mathcal{O}_{U,x}$よって
  \begin{equation*}
    I\mathcal{O}_{U,x}/J\mathcal{O}_{U,x} = I/J \otimes_{A} A_{\mathfrak{p}} = 0
  \end{equation*}
  が任意の$\mathfrak{p} \in \spec{A}$で成り立つ.今$A_{\mathfrak{p}}\neq 0$なので
  $I/J=0$で$I=J$は有限生成である.
}
\Definition{
  \index{あふぃんたようたい@アフィン(代数)多様体}\index{affine variety}
  \textbf{体$k$上のアフィン(代数)多様体(affine variety over $k$)}
  とは,$k$上有限生成代数に伴うアフィンスキームのことをいう.
  \index{だいすうたようたい@代数多様体}\index{algebraic variety}
  \textbf{体$k$上の代数多様体(algebraic variety over $k$)}とは,
  $k$-スキーム$X$であって,有限個のアフィン開集合$X_{i}$で被覆され,
  それぞれ$X_{i}$が$k$上のアフィン多様体であるときをいう.
  \index{しゃえいたようたい@射影(代数)多様体}\index{projective variety}
  \textbf{体$k$上の射影(代数)多様体(projective variety over $k$)}
  とは,$k$上の射影スキームのことをいう.
  射影多様体は代数多様体である.定義から$k$上の代数多様体の射は$k$-スキームとしての射である.
}

\Remark{
  代数多様体はネータースキームである.
}{}

\Remark{
  代数多様体$X$に対して$X^{0}$を$X$の閉点全体とする.包含写像
  \begin{equation*}
    i:X^{0} \hookrightarrow X
  \end{equation*}
  に対して$(X^{0},i^{*}\mathcal{O}_{X})$は局所環付き空間である.
}{}
