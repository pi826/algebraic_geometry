
\Section{Inductive Limit}
とりあえず,帰納極限だけ述べる.射影極限は双対概念なのでまぁ頑張って.
\Definition{(帰納系の定義)\\
($\Lambda,\leq$)を順序集合,$\mathscr{C}$を圏とする.各$\lambda \in \Lambda$に対し,$X_{\lambda} \in \text{Ob}(\mathscr{C})$が与えられ,
$\lambda \leq \mu$に対して射$\varphi_{\mu,\lambda}:X_{\lambda} \to X_{\mu}$があって
次を満たすとき,$\{X_{\lambda},\varphi_{\mu,\lambda}\}$を\textbf{順系(direct system)}または
\textbf{帰納系(inductive system)}という.しばし$\varphi_{\mu,\lambda}$を省略して$\{X_{\lambda}\}_{\lambda \in \Lambda}$や$\{X_{\lambda}\}_{\lambda}$で表す.
\begin{itemize}
  \item[---] 任意の$\lambda \in \Lambda$に対して$\varphi_{\lambda,\lambda}=\text{id}_{X_{\lambda}}$
  \item[---] $\lambda \leq \mu \leq \nu$なる任意の$\lambda,\mu,\nu \in \Lambda$に対して$\varphi_{\nu,\lambda} = \varphi_{\nu,\mu}\circ \varphi_{\mu,\lambda}$
\end{itemize}
}
\Example{
位相空間$X$の開集合族$\{U\}_{U}$に対して
\begin{equation*}
  U \leq V \defi V \subset U
\end{equation*}
と定義する.そして,$\mathbf{AGrp}$をアーベル群の成す圏,$\mathcal{F}$を$X$上の前層とする.すると,各開集合$U$に対し,$\mathcal{F}(U) \in \mathrm{Ob}(\mathbf{AGrp})$で,
前層の定義からアーベル群と制限写像との組$\{\mathcal{F}(U),\rho_{U,V}\}$は帰納系となる.前層の定義はDef:\ref{Def:1.3.1}を参照.
}{}
\Definition{(帰納系の射の定義)\\
$\Lambda$を順序集合.$\{X_{\lambda},\varphi_{\lambda,\mu}\},\{Y_{\lambda},\psi_{\lambda,\mu}\}$を$\Lambda$上の圏$\mathscr{C}$における帰納系とする.
このとき$\{X_{\lambda}\}$から$\{Y_{\lambda}\}$への射とは$f_{\lambda}:X_{\lambda} \to Y_{\lambda}$なる射の族
$\{f_{\lambda}\}$で,任意の$\lambda \leq \mu$に対して
$\psi_{\lambda,\mu}\circ f_{\mu} = f_{\lambda}\circ \varphi_{\lambda,\mu}$となるものを言う.
%---------------後に図式を追加する.------------------%
\begin{center}
\begin{tikzpicture}[auto]
  \node (X1) at (0,0) {$X_{\mu}$};
  \node (X2) at (0,-2.5) {$X_{\lambda}$};
  \node (Y1) at (2.5,0) {$Y_{\mu}$};
  \node (Y2) at (2.5,-2.5) {$Y_{\lambda}$};
  \node (ci) at (1.25,-1.25) {$\circlearrowleft$};

  \draw[->] (X1) to node[yshift = -6pt,label=above:$f_{\mu}$] () {} (Y1);
  \draw[->] (X1) to node[label=left:$\varphi_{\lambda,\mu}$] () {} (X2);
  \draw[->] (X2) to node[yshift = -2pt,label=below:$f_{\lambda}$] () {} (Y2);
  \draw[->] (Y1) to node[xshift = -6pt,label=right:$\psi_{\lambda,\mu}$] () {} (Y2);
\end{tikzpicture}
\end{center}
}
\Definition{
$\mathscr{C}$を圏とし,$\Lambda$を順序集合とする.$\{X_{\lambda},\varphi_{\mu,\lambda}\}$を
$\mathscr{C}$の帰納系とする.\\
このとき$\{X_{\lambda},\varphi_{\mu,\lambda}\}$の\textbf{順極限(direct limit)}または\textbf{帰納的極限(inductive limit)}または\textbf{帰納極限}とは,
$\mathscr{C}$の対象$\displaystyle \varinjlim_{\lambda \in \Lambda}X_{\lambda} \in \text{Ob}(\mathscr{C})$
と射の族$\displaystyle \{\varphi_{\lambda}:X_{\lambda} \to \varinjlim_{\lambda \in \Lambda}X_{\lambda}\}_{\lambda \in \Lambda}$の組
$\{\varinjlim X_{\lambda},\varphi_{\lambda}\}$で,次の条件を満たすものをいう.
\begin{itemize}
\item[---] $\lambda \leq \mu$に対して$\varphi_{\mu}\circ \varphi_{\mu,\lambda} = \varphi_{\lambda}$
\item[---] $\lambda \leq \mu$に対して$f_{\mu}\circ \varphi_{\mu,\lambda} = f_{\lambda}$を満たす任意の射の族$\{f_{\lambda}:X_{\lambda} \to Y\}_{\lambda \in \Lambda}$に対して,
      $\displaystyle f:\varinjlim_{\lambda \in \Lambda}X_{\lambda} \to Y$が一意に存在して
      \begin{equation*}
        f\circ \varphi_{\lambda} = f_{\lambda}\quad (\forall \lambda \in \Lambda)
      \end{equation*}
      を満たす.
\end{itemize}
}
\Remark{
一般の圏では帰納極限や射影極限は存在するとは限らない.しかし,存在するとすれば,同型を除いて一意である.
}{}
\Proposition{
帰納極限は存在すれば,同型を除いて一意である.
}{
証明は後で書く.
}
