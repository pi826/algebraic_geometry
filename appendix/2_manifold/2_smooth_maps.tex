
\Section{Smooth Maps}

\Definition{
  $n$次元可微分多様体$M$,$k\in \mathbf{Z}_{> 0}$に対して$f:M\to \mathbf{R}^{k}$
  が\index{なめらか@なめらか!なかんすう@なかんすう}\index{smooth! function}
  \textbf{滑らかな関数(smooth function)}とは,任意の点$p\in M$に対して,
  それを含む滑らかなチャート$(U,\varphi)$があって
  \begin{equation*}
    f\circ \varphi^{-1}:\varphi(U) \to \mathbf{R}^{k}
  \end{equation*}
  が滑らかなときをいう.
  \begin{equation*}
    C^{\infty}(M) \defi \{f:M\to \mathbf{R} \mid f \text{ is smooth}\}
  \end{equation*}
  と置く.これは$\mathbf{R}$上のベクトル空間である.
}

\Definition{
  $M,N$を可微分多様体とする.このとき$F:M\to N$が\index{なめらか@滑らか!なしゃぞう@な写像}\index{smooth! map}
  \textbf{滑らかな写像(smooth map)}とは任意の点$p\in M$に対して,それを含む
  滑らかなチャート$(U,\varphi)$と$F(p)$を含む滑らかなチャート$(V,\psi)$があって,
  $F(U)\subset V$で,
  \begin{equation*}
    \psi \circ F \circ \varphi^{-1}:\varphi(U) \to \psi(V)
  \end{equation*}
  が滑らかなときをいう.
}

\Proposition{
  滑らかな写像は連続である.
}{
  $M,N$を可微分多様体として,$F:M\to N$は$p\in M$で滑らかとする.
  つまり,$p$を含む滑らかなチャート$(U,\varphi)$と$F(p)$を含む滑らかなチャート$(V,\psi)$が$F(U)\subset V$で,
  \begin{equation*}
    \psi \circ F\circ \varphi^{-1} : \varphi(U) \to \psi(V)
  \end{equation*}
  が滑らかであるようにとれる.よって$\psi \circ F\circ \varphi^{-1}$は連続で,
  $\varphi :U \to \varphi(U)$,$\psi:V\to \psi(V)$は同相写像なので,
  \begin{equation*}
    F|_{U} = \psi^{-1} \circ (\psi \circ F\circ \varphi^{-1}) \circ \varphi : U\to V
  \end{equation*}
  は,連続である.よって任意の点の近傍で$F$は連続なので,$M$で連続である.
}

\Proposition{
  可微分多様体$M,N,P$に対して以下が成り立つ.
  \begin{itemize}
    \item[(1)] 任意の定数写像$c:M\to N$は滑らかである.
    \item[(2)] 恒等写像$\text{id}_{M}:M\to M$は滑らかである.
    \item[(3)] 開部分多様体$U\subset M$に対して包含写像$U \hookrightarrow M$は滑らかである.
    \item[(4)] $F:M\to N$,$G:N\to P$が滑らかなら,$G\circ F:M\to P$は滑らかである.   
  \end{itemize}
}{
  \Claim{(1)}\\
  $p\in M$を含む滑らかなチャート$(U,\varphi)$と,$c(p) = c$を含む滑らかなチャート$(V,\psi)$は$c(U) = \{c\} \subset V$である.また任意の$x\in \varphi(U)$に対して
  \begin{equation*}
    (\psi\circ c \circ \varphi^{-1})(x) = \psi(c)
  \end{equation*}
  で,$\psi$は$C^{\infty}$なので,定数写像$c$は滑らかである.\\
  \Claim{(2)}\\
  (1)と同じようにチャート$(U,\varphi)$,$(V,\psi)$を取る.ただし$\text{id}_{M}(U)=U\subset V$となるように取る.すると
  \begin{equation*}
    \psi \circ \text{id}_{M} \circ \varphi^{-1} = \psi \circ \varphi^{-1}
  \end{equation*}
  これは座標変換写像であり,定義から$C^{\infty}$である.よって$\text{id}_{M}$は滑らかである.\\
  \Claim{(3)}\\
  部分多様体の定義をかけ!!\\
  \Claim{(4)}\\
  $p\in M$を含む滑らかなチャート$(U,\varphi)$と,$F(p)$を含む滑らかなチャート$(V,\psi)$で$F(U)\subset V$が成り立つように取る.また,$G(F(p))$を含む滑らかなチャート$(W,\theta)$で,$G(V)\subset W$が成り立つように取る. 
  (つまり,$G(F(U))\subset W$が成り立つ.)よって,$F,G$の滑らかさから,
  \begin{align*}
    \psi \circ F \circ \varphi^{-1}:\varphi(U) \to \psi(V)\\
    \theta \circ G \circ \psi^{-1}:\psi(V) \to \theta(W)
  \end{align*}
  は滑らかである.よって,
  \begin{equation*}
    \theta \circ (G \circ F) \circ \varphi^{-1} = (\theta \circ G \circ \psi^{-1}) \circ (\psi \circ F \circ \varphi^{-1})
  \end{equation*}
  は滑らかである.
}
