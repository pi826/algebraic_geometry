
\Section{Smooth Maps}

\Definition{
  $n$次元可微分多様体$M$,$k\in \mathbf{Z}_{> 0}$に対して$f:M\to \mathbf{R}^{k}$
  が\index{なめらか@なめらか!なかんすう@なかんすう}\index{smooth! function}
  \textbf{滑らかな関数(smooth function)}とは,任意の点$p\in M$に対して,
  それを含む滑らかなチャート$(U,\varphi)$があって
  \begin{equation*}
    f\circ \varphi^{-1}:\varphi(U) \to \mathbf{R}^{k}
  \end{equation*}
  が滑らかなときをいう.
  \begin{equation*}
    C^{\infty}(M) \defi \{f:M\to \mathbf{R} \mid f \text{ is smooth}\}
  \end{equation*}
  と置く.これは$\mathbf{R}$上のベクトル空間である.
}

\Definition{
  $M,N$を可微分多様体とする.このとき$F:M\to N$が\index{なめらか@滑らか!なしゃぞう@な写像}\index{smooth! map}
  \textbf{滑らかな写像(smooth map)}とは任意の点$p\in M$に対して,それを含む
  滑らかなチャート$(U,\varphi)$と$F(p)$を含む滑らかなチャート$(V,\psi)$があって,
  $F(U)\subset V$で,
  \begin{equation*}
    \psi \circ F \circ \varphi^{-1}:\varphi(U) \to \psi(V)
  \end{equation*}
  が滑らかなときをいう.
}

\Proposition{
  滑らかな写像は連続である.
}{
  $M,N$を可微分多様体として,$F:M\to N$は$p\in M$で滑らかとする.
  つまり,$p$を含む滑らかなチャート$(U,\varphi)$と$F(p)$を含む滑らかなチャート$(V,\psi)$が$F(U)\subset V$で,
  \begin{equation*}
    \psi \circ F\circ \varphi^{-1} : \varphi(U) \to \psi(V)
  \end{equation*}
  が滑らかであるようにとれる.よって$\psi \circ F\circ \varphi^{-1}$は連続で,
  $\varphi :U \to \varphi(U)$,$\psi:V\to \psi(V)$は同相写像なので,
  \begin{equation*}
    F|_{U} = \psi^{-1} \circ (\psi \circ F\circ \varphi^{-1}) \circ \varphi : U\to V
  \end{equation*}
  は,連続である.よって任意の点の近傍で$F$は連続なので,$M$で連続である.
}
