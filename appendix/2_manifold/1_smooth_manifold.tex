
\Section{Smooth Manifold}

\Definition{
  位相空間$M$が\index{きょくしょゆーくりっどてき@局所ユークリッド的}\index{locally Euclidean}
  \textbf{$n$次元局所ユークリッド的(locally Euclidean of dimension $n$)}とは
  $M$の任意の点に対してある開近傍があって,これが$\mathbf{R}^{n}$のある開集合に同相なときをいう.
}

\Definition{
  位相空間$M$が\index{いそうたようたい@位相多様体}\index{topological manifold}
  \textbf{$n$次元位相多様体(topological manifold of dimension $n$/topological $n$-manifold)}
  とは,以下の条件を満たすときをいう.
  \begin{itemize}
    \item[(1)] $M$はHausdorffである.
    \item[(2)] $M$は第二可算である.
    \item[(3)] $M$は$n$次元局所ユークリッド的である.  
  \end{itemize}
  $M$の次元を$\dim M$とかく.
}

\Remark{
  位相多様体$M$の各点$p$に対してある開近傍$p \in U$があってある$\mathbf{R}^{\dim M}$
  の開集合$\widetilde{U}$があって,同相写像$\varphi:U\to \widetilde{U}$があるので
  $U$に対して$\varphi$を用いて$\widetilde{U}$の座標系を導入するようなことを考えよう.
}{}

\Definition{
  $n$次元位相多様体$M$に対して,\index{ちゃーと@チャート}\index{chart}
  組$(U,\varphi:U\to \widetilde{U})$を
  \textbf{$M$のチャート(coordinate chart/chart)}という.このとき$U$を
  座標開集合,$\varphi$を座標写像という.$x^{i}:M\to \mathbf{R}$で
  $\varphi = (x^{1},\dots,x^{n})$とする.$x^{1},\dots,x^{n}$を$U$の局所座標という.
  $\varphi(p) = 0$のとき,チャート$(U,\varphi)$は\textbf{点$p$を中心とするチャート}という.
}

\noindent
当たり前のことだが,点$p$を含むチャート$(U,\varphi)$
\footnote{点$p$を含むチャート$(U,\varphi)$とは$p\in U$のことをいう.}
に対して平行移動を施すことで,
$p$を中心とするチャートを作ることができる.
また$\varphi = (x^{1},\dots,x^{n})$とできるので$(U,x^{1},\dots,x^{n})$と書くことがある.

\Example{
  ユークリッド空間$\mathbf{R}^{n}$はチャート$(\mathbf{R}^{n},\text{id}_{\mathbf{R}^{n}})$で覆われている.
}{}

\Example{
  開集合$U\subset \mathbf{R}^{n}$上の連続関数$f:U\to \mathbf{R}^{k}$とする.
  このとき$f$のグラフ
  \begin{equation*}
    \Gamma(f) = \{(x,f(x)) \in \mathbf{R}^{n} \times \mathbf{R}^{k}\mid x\in U\}
  \end{equation*}
  これは$n$次元位相多様体になる.
}{
  グラフ$\Gamma(f)$は$\mathbf{R}^{n}\times \mathbf{R}^{k}$の部分空間なので,ハウスドルフで第二可算である.
  $\pi :\mathbf{R}^{n} \times \mathbf{R}^{k}\to \mathbf{R}^{n}$を第一成分の射影
  つまり$\pi(x,y) = x$とする.$\varphi:\Gamma(f) \to U$を
  $\pi$の$\Gamma(f)$への制限とする.
  \begin{equation*}
    \varphi(x,y) = x\qquad ((x,y) \in \Gamma(f))
  \end{equation*}
  このとき$\varphi$は連続写像であることに注意する.またこれは同相写像である.
  実際$\psi:U\subset \mathbf{R}^{n} \to \mathbf{R}^{n}\times\mathbf{R}^{k}$とすると
  \begin{equation*}
    \psi(x) = (x,f(x)) 
  \end{equation*}
  とすると,$\psi(U) = \Gamma(f)$,連続写像で
  \begin{align*}
    &(\varphi\circ \psi)(x) = \varphi(\psi(x)) = \varphi(x,f(x)) = x\\
    &(\psi \circ \varphi)(x,f(x)) = \psi(\varphi(x,f(x))) = \psi(x) = (x,f(x))
  \end{align*}
  つまり$\psi$は$\varphi$の逆写像を与え,これは連続なので$\varphi$は同相写像である.
  つまり$\Gamma(f)$はチャート$(\Gamma(f),\varphi)$で覆われた$n$次元位相多様体である.
}

\Example{
  $n\in \mathbf{N}$に対して$n$次元球面$\mathbb{S}^{n}$
  \begin{equation*}
    \mathbb{S}^{n} = \{(x^{1},\dots ,x^{n+1}) \in \mathbf{R}^{n+1}\mid (x^{1})^2 + \dots + (x^{n+1})^2 = 1\}
  \end{equation*}
  は,$n$次元位相多様体になる.
}{
  まず$\mathbb{S}^{n}$は$\mathbf{R}^{n+1}$の部分空間なので,ハウスドルフ,第二可算である.
  次に局所ユークリッド的であることを示そう.
  \begin{equation*}
    U_{i}^{+} = \{(x^{1},\dots,x^{n+1}) \in \mathbf{R}^{n+1} \mid x^{i} > 0\}\qquad (i = 1,\dots,n+1)
  \end{equation*}
  と定義する,同様に$U_{i}^{-}$を$x^{i} < 0$なる$\mathbf{R}^{n+1}$の部分集合とする.
  連続写像$f:\mathbb{B}^{n} \to \mathbf{R}$を
  \begin{equation*}
    f(u) = \sqrt{1 - |u|^2}
  \end{equation*}
  で定義する.ここで$\mathbb{B}^{n}$は$n$次元球体である.
  \begin{equation*}
    \mathbb{B}^{n} = \{(x^{1},\dots,x^{n})\in \mathbf{R}^{n}\mid (x^{1})^2 + \dots + (x^{n})^2 < 1\}
  \end{equation*}
  $i = 1,\dots,n+1$に対して$U_{i}^{+}\cap \mathbb{S}^{n}$は連続関数のグラフである.
  \begin{equation*}
    x^{i} = f(x^{1},\dots,\widehat{x^{i}},\dots,x^{n+1})
  \end{equation*}
  $\widehat{x^{i}}$は$x^{i}$を除いて,という意味である.同様に$U_{i}^{-}\cap \mathbb{S}^{n}$は
  \begin{equation*}
    x^{i} = -f(x^{1},\dots,\widehat{x^{i}},\dots,x^{n+1})
  \end{equation*}
  のグラフである.すなわち,$U_{i}^{\pm}\cap \mathbb{S}^{n}$は$n$次元局所ユークリッド的である.
  写像$\varphi_{i}^{\pm}:U_{i}^{\pm}\cap \mathbb{S}^{n} \to \mathbb{B}^{n}$を
  \begin{equation*}
    \varphi_{i}^{\pm}(x^{1},\dots,x^{n+1}) = (x^{1},\dots,\widehat{x^{i}},\dots,x^{n+1})
  \end{equation*}
  で定義する.
}

一応関数$\mathbf{R}^{n}\to \mathbf{R}^{m}$が滑らかであることの定義しておこう.
\Definition{
  関数$F:U\subset \mathbf{R}^{n} \to V\subset \mathbf{R}^{m}$\index{なめらか@滑らか}\index{smooth}
  \index{しーむげん@$C^{\infty}$}\index{C infinity@$C^{\infty}$}
  が\textbf{滑らか(smooth)}もしくは\textbf{$C^{\infty}$}とは何回でも偏微分可能なときをいう.
  さらに$F$の逆写像が滑らかなとき,\index{びぶんどうそうしゃぞう@微分同相写像}\index{diffeomorphism}
  \textbf{微分同相写像(diffeomorphism)}という.特に微分同相写像は同相写像である
}
$n$次元位相多様体$M$の点$p\in M$の座標写像$\varphi:U\to \widetilde{U}\subset \mathbf{R}^{n}$
をとる.このとき$f:M\to \mathbf{R}$が点$p$で滑らかとは,$f\circ \varphi^{-1}:\widetilde{U}\subset \mathbf{R}^{n} \to \mathbf{R}$
が滑らかであることとする.ただ,この定義がwell-definendであるためには座標関数に
依らないことをいわなければならない.例えば,別の座標写像$\psi:V \to \widetilde{V}\subset \mathbf{R}^{n}$
をとる.このとき$f\circ \varphi^{-1} = f\circ \psi^{-1}$
\footnote{ここで左辺の定義域は$\widetilde{U} = \varphi(U)$で右辺の定義域は$\widetilde{V} = \psi(V)$で互いに異なる.ここでこの等号は定義域を$\varphi(U)\cap \psi(V)$に制限して考えている}
でなければならない.
これを定義するために色々準備しよう.
\Definition{
  $n$次元位相多様体$M$の二つのチャート$(U,\varphi),(V,\psi)$が$U\cap V \neq \varnothing$
  のとき,写像
  \begin{equation*}
    \psi\circ \varphi^{-1} : \varphi(U\cap V) \to \psi(U\cap V)
  \end{equation*}
  または
  \begin{equation*}
    \varphi \circ \psi^{-1} : \psi(U\cap V) \to \varphi(U\cap V)
  \end{equation*}
  を\index{ざひょうへんかんしゃぞう@座標変換写像}\index{transition map}
  \textbf{座標変換写像(transition map)}
  \footnote{$\psi\circ \varphi^{-1}$は正確には$\psi \circ (\varphi|_{U\cap V})^{-1}$のことである.}
  という.これは同相写像の合成なので同相写像である.二つのチャート$(U,\varphi),(V,\psi)$
  が$U\cap V = \varnothing$であるか,座標変換写像が微分同相であるとき
  \index{smoothly compatible}
  \textbf{smoothly compatible}という.
}

\begin{center}
\begin{tikzpicture}
  \draw[step = 0.5, thin, dotted] (-3,-3) grid (0,0) node[left] at (-3,-3.2) {$\mathbf{R}^{n}$};
  \draw[->] (-3,-3) -- (0,-3);
  \draw[->] (-3,-3) -- (-3,0);

  \draw[step = 0.5, thin, dotted] (3,-3) grid (6,0) node[left] at (3,-3.2) {$\mathbf{R}^{n}$};
  \draw[->] (3,-3) -- (6,-3);
  \draw[->] (3,-3) -- (3,0);

  \draw[thick] (-3,3) to[closed,curve through = {(-1,4) (1,5) (2,5) (3,4) (6,4) (4,1.5)}] (2,2) node[left] at (-3,4) {$M$};
  
  \draw[thick, dotted] (-1,3.5) to[closed,curve through = {(0,4) (2,4.5) (3,3.2)}] (2,2.5) node[left] at (-1.2,3) {$U$}; 


  \draw[thick, dotted] (1,4) to[closed,curve through = {(3,3.5) (4,3.7) (5,3)}] (2,2.5) node[right] at (5.2,3) {$V$};

  \begin{scope}
    \clip (-1,3.5) to[closed,curve through = {(0,4) (2,4.5) (3,3.2)}] (2,2.5);
    \fill[red, opacity = 0.2] (1,4) to[closed,curve through = {(3,3.5) (4,3.7) (5,3)}] (2,2.5);
  \end{scope}

  %\fill (2,3) circle (0.5mm);

  \draw[thick,dotted] (-2.5,-2.5) to[closed,curve through = {(-2.5,-1) (-1.5,0)}] (-1,-0.5) node[above] at (-2,0) {$\varphi(U)$};

  \draw[thick,dotted] (4,-2) to[closed,curve through = {(3.5,-1) (5.5,-2)}] (4.5,-2.5) node[above] at (5,0) {$\psi(V)$};

  % \draw[thick,dotted] (0,0) to[closed,curve through = {(-2,-1.5) (-1.5,-2)}] (0,-3);

  \begin{scope}
    \clip (-2.5,-2.5) to[closed,curve through = {(-2.5,-1) (-1.5,0)}] (-1,-0.5);
    \draw[thick,dotted] (0,0) to[closed,curve through = {(-2,-1.5) (-1.5,-2)}] (0,-3);
    \fill[red, opacity = 0.2] (0,0) to[closed,curve through = {(-2,-1.5) (-1.5,-2)}] (0,-3);
  \end{scope}

  %\draw[thick,dotted] (2.5,0) to[closed,curve through = {(4,-0.5) (5,-2)}] (3,-3);

  \begin{scope}
    \clip (4,-2) to[closed,curve through = {(3.5,-1) (5.5,-2)}] (4.5,-2.5);
    \draw[thick,dotted] (2.5,0) to[closed,curve through = {(4,-0.5) (5,-2)}] (3,-3);
    \fill[red, opacity = 0.2] (2.5,0) to[closed,curve through = {(4,-0.5) (5,-2)}] (3,-3);
  \end{scope}

  \draw[->,thick] (0,2) to[bend right = 25] (-1.2,0.2) node[right] at (-0.7,1) {$\varphi$};

  \draw[->,thick] (3,2) to[bend left = 25] (4.2,0.2) node[left] at (3.8,1) {$\psi$};

  \draw[->,thick] (-0.5,-0.5) to[bend left = 20] (3.2,-0.5) node[above] at (1.5,0) {$\psi\circ \varphi^{-1}$};

  \draw[<-,thick] (-0.5,-2.5) to[bend right = 20] (3.2,-2.5) node[below] at (1.5,-2) {$\varphi \circ \psi^{-1}$};
\end{tikzpicture}
\end{center}

\Definition{
  位相多様体$M$の\index{あとらす@アトラス}\index{atlas}
  \textbf{アトラス(atlas)}とは,$M$を被覆するチャートの族のことである.
  アトラス$\mathcal{A}$が\index{なめらか@滑らか!なあとらす@なアトラス}\index{smooth! atlas}
  \textbf{滑らかなアトラス(smooth atlas)}とは
  $\mathcal{A}$の任意の二つのチャートがsmoothly compatibleであるときをいう.
}
この滑らかなアトラスが位相多様体に微分構造(smooth structure)を定義するのだが,
同じ位相多様体に異なる微分構造が入った時いつ同一視するか?という問題がある.
例えば,$\mathbf{R}^{n}$のアトラス
\begin{align*}
  &\mathcal{A}_{1} = \{(\mathbf{R}^{n},\text{id}_{\mathbf{R}^{n}})\}\\
  &\mathcal{A}_{2} = \{(B(x;1),\text{id}_{B(x;1)})\}_{x\in \mathbf{R}^{n}}
\end{align*}
ここで$x\in \mathbf{R}^{n}$と正の実数$r$に対して
\begin{equation*}
  B(x;r) \defi \{y\in \mathbf{R}^{n}\mid d(x,y) < r\}
\end{equation*}
$d:\mathbf{R}^{n}\times \mathbf{R}^{n} \to \mathbf{R}_{\geq 0}$はユークリッド空間の通常の距離関数である.
これらは違う滑らかなアトラスだが,関数$f:\mathbf{R}^{n} \to \mathbf{R}$\\ \\
後で加筆する.

\Definition{
  滑らかなアトラス$\mathcal{A}$が\textbf{極大(maximal)}とは,$\mathcal{A}$を真に包含する
  滑らかなアトラスが存在しない時をいう.
}

\Definition{
  位相多様体$M$の極大な滑らかなアトラスを\index{びぶんこうぞう@微分構造}\index{smooth! structure}
  \textbf{微分構造(smooth structure)}という.\index{なめらか@滑らか!なたようたい@な多様体}\index{smooth! manifold}
  \textbf{滑らかな多様体(smooth manifold)}とは位相多様体とその微分構造の組$(M,\mathcal{A})$
  のことをいう.このとき微分構造が明らかなときは単に$M$は滑らかな多様体という.
  滑らかな多様体は,また\textbf{可微分多様体(differentiable manifold)}
  という.
}


\Proposition{
  位相多様体$M$に対して以下が成り立つ.
  \begin{itemize}
    \item[(1)] 任意の$M$の滑らかなアトラス$\mathcal{A}$に対して$\mathcal{A}$を含む極大な滑らかなアトラスが唯一存在する.これを\textbf{$\mathcal{A}$から決まる微分構造(smooth structure determined by $\mathcal{A}$)}という.
    \item[(2)] $M$の二つの滑らかなアトラスが同じ微分構造を決めるということはそれらの合併が滑らかなアトラスになることと必要十分である. 
  \end{itemize}
}{
  $\mathcal{A}$を$M$の滑らかなアトラス,$\overline{\mathcal{A}}$を
  $\mathcal{A}$の任意のチャートとsmoothly compatibleなチャートの集合とする.
  このとき$\overline{\mathcal{A}}$が滑らかなアトラスであることを示そう.
  つまり任意の$\overline{\mathcal{A}}$の二つのチャートがsmoothly compatible
  であることを示せば良い.つまり定義から$(U,\varphi),(V,\psi)\in \overline{\mathcal{A}}$
  に対して
  \begin{equation*}
    \psi\circ \varphi^{-1}:\varphi(U\cap V) \to \psi(U\cap V)
  \end{equation*}
  が滑らかであることを確認すればよい.$\varphi(p) \in \varphi(U\cap V)$
  をとる.アトラス$\mathcal{A}$は$M$を被覆するので$p\in W$なる$(W,\theta) \in \mathcal{A}$
  を取れる.$\overline{A}$の任意のチャートは$(W,\theta)$とsmoothly compatible
  なので,$\theta\circ \varphi^{-1}$,$\psi \circ \theta^{-1}$は滑らかである.
  よって,$p\in U\cap V \cap W$に対して,
  \begin{equation*}
    \psi \circ \varphi^{-1} = (\psi \circ \theta^{-1})\circ (\theta \circ \varphi^{-1})
  \end{equation*}
  は点$\varphi(p)$近傍で滑らかである.今$\varphi(p)\in \varphi(U\cap V)$は任意なので
  $\psi\circ \varphi^{-1}$は$\varphi(U\cap V)$の任意の点の近傍で滑らかである.
  よって$\overline{\mathcal{A}}$は滑らかなアトラスである.\\
  あとは,極大であることを確認すれば良い.$\overline{\mathcal{A}}\subset \mathcal{B}$なる滑らかなアトラス$\mathcal{B}$を取ると,
  $\mathcal{A} \subset \mathcal{B}$と定義から$\mathcal{B}$の任意のチャートは
  $\mathcal{A}$の任意のチャートとsmoothly compatibleである.従って$\overline{\mathcal{A}}$の取り方から逆の包含関係も分かる.又,同様に,唯一性もわかる.
}

smoothly compatibleを弱めて,アトラスの任意の二つのチャートの座標変換写像が$C^{k}$級
で,極大なとき\textbf{$C^{k}$構造($C^{k}$ structure)}という.同様に,座標変換写像が,
実解析的(real-analytic)なとき,\textbf{実解析的構造(real-analytic structure)}
または\textbf{$C^{\omega}$構造($C^{\omega}$ structure)}という.
もし次元が$2m$なら$\mathbf{R}^{2m}$と$\mathbf{C}^{m}$とを同一視して,また,
座標変換写像を複素解析的(正則関数)とすることで,\textbf{複素解析的構造(complex-analytic structure)}が定義できる.


\Definition{
  可微分多様体$M$のチャート$(U,\varphi)$が$M$の微分構造(滑らかなアトラス)に含まれているとき\index{なめらか@滑らか!なちゃーと@なチャート}\index{smooth! chart}
  \textbf{滑らかなチャート(smooth chart)}という.また,$\varphi$を
  \index{なめらか@滑らか!なざひょうしゃぞう@な座標写像}\index{smooth! coordinate map}
  \textbf{滑らかな座標写像(smooth coordinate map)}という.
  \index{なめらか@滑らか!なざひょうきゅう@な座標球}\index{smooth! coordinate ball}
  \textbf{滑らかな座標球(smooth coordinate ball)}とは,滑らかのチャート$(U,\varphi)$
  であり,$\varphi(U)$がユークリッド空間の球であるときをいう.同様に
  \index{なめらか@滑らか!なざひょうりっぽうたい@な座標立方体}\index{smooth! coordinate cube}
  \textbf{滑らかな座標立方体(smooth coordinate cube)}が定義される.
}

\Definition{
  $B$が\index{せいそくざひょうきゅう@正則座標球}\index{regular coordinate ball}
  \textbf{正則座標球(regular coordinate ball)}とは$\overline{B}\subset B'$なる
  滑らかな座標球$B'$が存在し,滑らかな座標写像$\varphi:B'\to \mathbf{R}^{n}$が
  ある$r < r'$なる正の実数$r,r'$に対して
  \begin{equation*}
    \varphi(B) = B(0;r),\ \varphi(\overline{B}) = \overline{B(0;r)},\ \varphi(B') = B(0;r')
  \end{equation*}
  となるときをいう.
}

\Proposition{
  任意の可微分多様体は正則座標球の加算開基を持つ.
}{}



