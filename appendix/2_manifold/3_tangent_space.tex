
\Section{Tangent Space}

まず,可微分多様体$M$と$p\in M$に対して,$p$近傍で定義されている$C^{\infty}$実数値関数$f,g$に対して二項関係$\sim$を
\begin{equation*}
  f\sim g \defi p\in \exists W:\text{open}, f|_{W} = g|_{W}
\end{equation*}
と定義する.すると,これは同値関係である.この同値関係で$p$近傍で定義されている$C^{\infty}$実数値関数の集合を割った集合を$C^{\infty}_{p}(M)$と書く.
\begin{equation*}
  C^{\infty}_{p}(M) \defi \{f:U\to \mathbf{R}\mid p\in U:\text{open}\}/\sim
\end{equation*}
これは,関数の和と積で$\mathbf{R}$多元環を成す.\\
$C^{\infty}_{p}(M)$の元は本来$C^{\infty}$関数$f:U\to \mathbf{R}$を用いて,
\begin{equation*}
  [f] \in C^{\infty}_{p}(M)
\end{equation*}
などと書かれるべきだが,面倒なのでこれを$f$と書いてしまおう.\\
さて,早速接空間(tangent space)を定義していこう.
\Definition{
  可微分多様体$M$の\index{どうぶん@導分}\index{derivation}
  \textbf{点$p$における導分(derivation at $p$)}とは,線形写像$D:C^{\infty}_{p}(M) \to \mathbf{R}$であって,
  \begin{equation*}
    D(fg) = D(f)g(p) + f(p)D(g)
  \end{equation*}
  を満たすものとして定義する.また,$M$の点$p$における\index{せつべくとる@接ベクトル}\index{tangent vector}
  \textbf{接ベクトル(tangent vector)}とは,点$p$における導分のことである.\\
  点$p$における接ベクトル全体はベクトル空間を成す.これを$T_{p}(M)$または,$T_{p}M$と書いて,\index{せつくうかん@接空間}\index{tangent space}\textbf{点$p$における$M$の接空間(tangent space to $M$ at $p$)}という.
}

