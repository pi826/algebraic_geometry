
\subsection{differential}

\Definition{
  可微分多様体$M,N$の間の$C^{\infty}$写像$F:M\to N$を取る.各点$p\in M$に対して
  $F$の\index{びぶん@微分}\index{differential}
  \textbf{点$p$における微分(differential at $p$)}と呼ばれる接空間の間の線型写像
  \begin{equation*}
    dF:T_{p}M \to T_{F(p)}N
  \end{equation*}
  が,$X_{p}\in T_{p}M$に対して$dF(X_{p})$は
  \begin{equation*}
    dF(X_{p})(f) \defi X_{p}(f\circ F)\qquad (f\in C^{\infty}_{F(p)}(N))
  \end{equation*}
  なる接ベクトルである.
}

\Remark{
  どの点における微分であるかを明確にするために$dF$の代わりに$dF_{p}$と書くことがある.
}{}

滑らかな写像$F:M\to N$,$G:N\to P$と点$p\in M$に対して線型写像
\[
  \xymatrix{
    T_{p}M \ar@/^25pt/@{>}[rr]^{dG_{F(p)}\circ dF_{p}} \ar@/_25pt/@{>}[rr]_{d(G\circ F)_{p}} \ar[r]^{dF_{p}}& T_{F(p)}N \ar[r]^{dG_{F(p)}} &T_{G(F(p))}
  }
\]

が定義できる.これらは等しい.
\Theorem{
  $F:M\to N$,$G:N\to P$を可微分多様体の間の滑らかな写像とする.このとき$p\in M$ならば
  \begin{equation*}
    d(G\circ F)_{p} = dG_{F(p)}\circ dF_{p}
  \end{equation*}
  が成り立つ.
}{
  $X_{p}\in T_{p}M$を取る.$f\in C^{\infty}_{G(F(p))}(P)$に対して,
  \begin{equation*}
    d(G\circ F)(X_{p})(f) = X_{p}(f\circ (G\circ F))
  \end{equation*}
  で,また
  \begin{equation*}
    (dG\circ dF)(X_{p})(f) = dG(dF(X_{p}))(f) = dF(X_{p})(f\circ G) = X_{p}(f\circ G\circ F)
  \end{equation*}
}

\Corollary{
  $F:M\to N$が可微分多様体の間の微分同相写像とする.このとき$dF_p:T_{p}M \to T_{F(p)}N$は同型写像である.
}{}

\Corollary{
  開集合$U\subset \mathbf{R}^{n}$が開集合$V\subset \mathbf{R}^{m}$と微分同相ならば,$n=m$である.
}{
  $F:U\to V$を微分同相写像とする.このとき先程の系より,
  \begin{equation*}
    dF_{p} : T_{p}U \to T_{F(p)}V
  \end{equation*}
  は同型写像である.また,$T_{p}U = T_{p}\mathbf{R}^{n} \simeq \mathbf{R}^{n}$
  ,$T_{F(p)}V = T_{F(p)}\mathbf{R}^{m} \simeq \mathbf{R}^{m}$より,$n=m$を得る.
}

$n$次元可微分多様体$M$の点$p$を含む滑らかなチャート$(U,\varphi = (x^{1},\dots ,x^{n}))$をとる.このとき$\varphi:U\to \varphi(U)$は微分同相写像なので,
\begin{equation*}
  d\varphi_{p}:T_{p}M\to T_{\varphi(p)}\mathbf{R}^{n}
\end{equation*}
は同型写像である.特に$T_{p}M$の次元は$n$である.

\Proposition{
  $(U,\varphi = (x^{1},\dots ,x^{n}))$を$n$次元可微分多様体$M$の点$p$を含む滑らかなチャートとする.このとき
  \begin{equation*}
    d\varphi_{p}\left( \left. \frac{\partial}{\partial x^{i}} \right|_{p}\right) = \left. \frac{\partial}{\partial r^{i}} \right|_{\varphi(p)}
  \end{equation*}
}{
  $f\in C^{\infty}_{\varphi(p)}(\mathbf{R}^{n})$に対して
  \begin{align*}
    d\varphi_{p}\left( \left. \frac{\partial}{\partial x^{i}} \right|_{p}\right)f
    &= \left. \frac{\partial}{\partial x^{i}} \right|_{p} (f\circ \varphi)\\
    &= \left. \frac{\partial}{\partial r^{i}} \right|_{\varphi(p)} (f\circ \varphi \circ \varphi^{-1})\\
    &= \left. \frac{\partial}{\partial r^{i}} \right|_{\varphi(p)} f
  \end{align*}
}

\Proposition{
  点$p\in M$を含む滑らかなチャート$(U,\varphi = (x^{1},\dots ,x^{n}))$を取る.すると接空間$T_{p}M$は基底
  \begin{equation*}
    \left. \frac{\partial}{\partial x^{1}} \right|_{p},\dots ,\left. \frac{\partial}{\partial x^{n}} \right|_{p}
  \end{equation*}
  を持つ.
}{
  ベクトル空間の同型写像は基底を基底に移す.先程の命題より,同型写像
  \begin{equation*}
    d\varphi_{p}:T_{p}M \to T_{\varphi(p)}\mathbf{R}^{n}
  \end{equation*}
  は$\partial/\partial x^{i}|_{p}$を$\partial/\partial r^{i} |_{\varphi(p)}$に移すが,
  $\partial/\partial r^{i}|_{\varphi(p)}$は$T_{\varphi(p)}\mathbf{R}^{n}$の基底なので,$\partial/\partial x^{i}|_{p}$は$T_{p}M$の基底である.
}

