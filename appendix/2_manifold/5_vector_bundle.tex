
\Section{Vector Bundle}

底空間を位相空間とする場合でベクトルバンドルを定義しよう.

\Definition{
  $X$を位相空間とする.$X$上の\textbf{ベクトル空間の族(family of vector spaces)}とは,位相空間$E$に
  \begin{itemize}
    \item[(1)] 連続写像$p:E\to X$
    \item[(2)] $x\in X$に対するそれぞれの
    \begin{equation*}
      E_{x} = p^{-1}(\{x\})
    \end{equation*}
    が有限次元ベクトル空間であり,$E$から誘導される位相と整合的である.
  \end{itemize}
  を満たすものをいう.
}
族$p:E\to X$の\textbf{切断(section)}とは連続写像$s:X\to E$で,$p\circ s = 1$となるときをいう.

