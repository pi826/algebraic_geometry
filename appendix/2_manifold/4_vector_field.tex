
\Section{Vector Field}
多様体$M$においてそのすべての点で時間$t$に応じて運動し
$M$の上に滑らかな流れが生じている状態を考えてみよう.

\begin{center}
\begin{tikzpicture}
  \clip (-3,3) to[closed, curve through = {(-2,-1) (2,-2.5) (2,2.5)}] (0,1) node[left] at (-3,4) {$M$};

% ベクトル場の描画
\foreach \x in {-4,-3.5,...,5} {
    \foreach \y in {-4,-3.5,...,4} {
        % ベクトルの方向と長さを定義
        \pgfmathsetmacro{\vx}{-\y}
        \pgfmathsetmacro{\vy}{\x}
        \pgfmathsetmacro{\magnitude}{sqrt(\vx*\vx+\vy*\vy)}
        % ゼロ除算を防ぐ
        \ifdim \magnitude pt > 0pt
            \pgfmathsetmacro{\scale}{0.3/\magnitude}
        \else
            \pgfmathsetmacro{\scale}{0} % 矢印を描画しない
        \fi
        % 矢印の描画(\scaleがゼロの場合はスキップ)
        \ifdim \scale pt > 0pt
            \draw[->, blue] (\x, \y) -- ++(\scale*\vx, \scale*\vy);
        \fi
    }
}




\draw[thick] (-3,3) to[closed, curve through = {(-2,-1) (2,-2.5) (2,2.5)}] (0,1) node[left] at (-3,4) {$M$};



\end{tikzpicture}
\end{center}




\Definition{
多様体$M$において,各実数$t$に対して滑らかな写像$\varphi_{t}:M \to M$
が与えられ,次を満たすとき$\{\varphi_{t}\}$を$M$の\textbf{1パラメーター変換群}という.
\begin{itemize}
  \item[(1)] $\varphi_{s}\circ \varphi_{t} = \varphi_{s+t},\quad \varphi_{0} = 1_{M}$
  \item[(2)] 写像$\mathbf{R}\times M\to M;(t,x)\mapsto \varphi_{t}(x)$は滑らかである. 
\end{itemize}
}
曲線の接ベクトルを考えれば
\begin{equation}
  X_{x}f = \lim_{t\to 0}\frac{f(\varphi_{t}(x)) - f(x)}{t}
\end{equation}
