
\Section{Several Variables}
$\mathbf{C}^{n}$の領域$D$上で定義された関数$f$を考える.$z = (z_{1},\dots,z_{n}) \in \mathbf{C}^{n}$をとる.\\
$x_{j} = \Re z_{j},\ y_{j} = \Im z_{j}\ (j = 1,\dots ,n)$として,
一変数の場合と同様に,
\begin{align}
  \frac{\partial}{\partial z_{j}} =\frac{1}{2}\left( \frac{\partial}{\partial x_{j}} + \frac{1}{i}\frac{\partial}{\partial y_{j}} \right)\\
  \frac{\partial}{\partial \overline{z_{j}}} =\frac{1}{2}\left( \frac{\partial}{\partial x_{j}} - \frac{1}{i}\frac{\partial}{\partial y_{j}}\right)
\end{align}
と定義する.$f$が$(x_{1},y_{1},x_{2},y_{2},\dots,x_{n},y_{n})$について$C^{1}$級で
$D$上で
\begin{equation}
  \frac{\partial f}{\partial \overline{z_{j}}} = 0 \qquad (j = 1,\dots,n)
\end{equation}
を満たすとき$f$は$D$で\textbf{正則(holomorphic)}という.あるいは$f$は$D$上の
\textbf{正則関数(holomorphic function)}と呼ばれる.\\
次に一次元の円板$D\subset \mathbf{C}$に相当する多重円板を定義する.\\
$c = (c_{1},\dots,c_{n}) \in \mathbf{C}^{n}$,$r= (r_{1},\dots,r_{n}) \in (\mathbf{R}_{>0})^{n}$に対して,
\begin{equation}
  \Delta(c;r) = \{z = (z_{1},\dots,z_{n}) \in \mathbf{C}^{n} \mid |z_{j} - c_{j}| < r_{j},\ j = 1,\dots ,n\} = D(c_{1};r_{1})\times \dots \times D(c_{n};r_{n})
\end{equation}
の形の領域を$c$を中心とする\textbf{多重円板(polydisc/polydisk)}という.\\
$f(z) = f(z_{1},\dots ,z_{n})$は多重円板$\Delta(c;r)$の閉包$\overline{\Delta(c;r)}$を
含む領域で正則な関数とする.$(z_{2},\dots,z_{n})$を固定して$z_{1}$に関して
一変数のCauchyの積分公式を用いれば,$(z_{1},\dots,z_{n}) \in \Delta$に対して
\begin{equation}
  f(z_{1},\dots,z_{n}) = \frac{1}{2\pi i}\int_{D(c_{1};r_{1})}\frac{f(\zeta_{1},z_{2},\dots,z_{n})}{\zeta_{1} - z_{1}}d\zeta_{1}
\end{equation}
である.ただし円板$D(c_{1};r_{1})$は正の向きに向き付けられているとする.
これを$z_{2},\dots,z_{n}$に繰り返し適用すれば,
\begin{align*}
  f(z_{1},\dots,z_{n}) 
  &= \frac{1}{2\pi i}\int_{D(c_{1};r_{1})}\frac{f(\zeta_{1},z_{2},\dots,z_{n})}{\zeta_{1} - z_{1}}d\zeta_{1}\\
  &= \left(\frac{1}{2\pi i}\right)^{2}\int_{D(c_{1};r_{1})}\int_{D(c_{2};r_{2})}\frac{f(\zeta_{1},\zeta_{2},z_{3},\dots,z_{n})}{(\zeta_{2} - z_{2})(\zeta_{1} - z_{1})}d\zeta_{2}d\zeta_{1}\\
  &= \left(\frac{1}{2\pi i}\right)^{n}\int_{D(c_{1};r_{1})}\dots \int_{D(c_{n};r_{n})}\frac{f(\zeta_{1},\dots,\zeta_{n})}{(\zeta_{n} - z_{n})\dots (\zeta_{1} - z_{1})}d\zeta_{n} \dots d\zeta_{1}\\
\end{align*}
と書くことが出来る.\\
$n$変数の場合,$c = (c_{1},\dots,c_{n}) \in \mathbf{C}^{n}$における冪級数は
\begin{equation}
  \sum_{m_{1} = 0}^{\infty}\dots \sum_{m_{n} = 0}^{\infty}a_{m_{1},\dots,m_{n}}(z_{1} - c_{1})^{m_{1}}\dots (z_{n} - c_{n})^{m_{n}}
\end{equation}
である.一変数の場合と同様にして,$\mathbf{C}^{n}$の領域$D$で正則な関数$f$は
$D$の各点$c$の近傍で絶対収束する冪級数に展開される.このとき展開の係数は
\begin{align}
  a_{m_{1},\dots,m_{n}} 
  &= \frac{1}{m_{1}!\dots m_{n}!}\frac{\partial^{m_{1} + \dots + m_{n}}f}{\partial z_{1}^{m_{1}}\dots \partial z_{n}^{m_{n}}}(c)\\
  &= \left(\frac{1}{2\pi i}\right)^{n}\int_{D(c_{1};r_{1})}\dots \int_{D(c_{n};r_{n})}\frac{f(\zeta_{1},\dots,\zeta_{n})}{(\zeta_{n} - c_{n})^{m_{n} + 1}\dots (\zeta_{1} - c_{1})^{m_{1} + 1}}d\zeta_{n}\dots d\zeta_{1}
\end{align}
で与えられる.したがって$D(c_{1};r_{1})\times \dots \times D(c_{n};r_{n})$における
$|f(z_{1},\dots,z_{n})|$の最大値を$M$とすれば,
\begin{equation}
  |a_{m_{1},\dots,m_{n}}|\leq \frac{M}{r_{1}^{m_{1}}\dots r_{n}^{m_{n}}}
\end{equation}
が成り立つ.特に$r_{1},\dots,r_{n}$の最小値を$r$とすれば,
\begin{equation}
  |a_{m_{1},\dots,m_{n}}|\leq \frac{M}{r^{m_{1}+\dots + m_{n}}}
\end{equation}
が成り立つ.
一変数の場合と同様に,これらの結果を用いて一致の定理と最大値の原理が証明される.
\Theorem{(Hartogs' Theorem)
  $D$を$\mathbf{C}^{n}$の原点$0$を含む領域で$n\geq 2$とする.$f$が$D\mysetminus \{0\}$
  で定義された正則関数ならば$f$は$D$上の正則関数に拡張される.
}{
  $D$は多重円板$\{z \in \mathbf{C}^{n}\mid |z_{j}| < r_{j},\ j = 1,\dots,n\}$
  としてよい.正数$s$を$0 < s < r_{1}$となるように選び
  \begin{equation}
    g(z_{1},\dots,z_{n}) = \frac{1}{2\pi i}\int_{D(0;s)}\frac{f(\zeta_{1},z_{2},\dots,z_{n})}{\zeta_{1} - z_{1}}d\zeta_{1}
  \end{equation}
  と定義する.ここで$D(0;s)$は正の向きに向き付けられているとする.このとき
  $g$は$D' = \{z \in \mathbf{C}^{n} \mid |z_{1}| < s,\ |z_{j}| < r_{j},\ j = 2,\dots,n\}$
  で正則な関数を定める.
}
