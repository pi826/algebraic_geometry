
\Section{Cauchy's Integral Theorem}
コーシーの積分定理とは,以下のことである.\\
$D$を領域,$f$を$D$上の正則関数,$C$を$D$内の閉曲線で$D$内で$C$が$0$にホモトープ($C\simeq 0$)ならば
\begin{equation}
  \int_{C}f(z)dz = 0
\end{equation}
が成り立つ.\\
ホモトープの定義などは後で述べる.最初に$C$が三角形の辺のなす折れ線の場合を示す.
$\mathbf{C}$の三点$a,b,c$を頂点とする閉三角形を$T(a,b,c)$で表す.
\begin{equation}
  T(a,b,c) = \{z = \lambda a + \mu b + \nu c \in \mathbf{C} \mid \lambda + \mu + \nu = 1,\ \lambda, \mu,\nu, \geq 0\}
\end{equation}
と表される.$T$の内部は
\begin{equation}
  \{z = \lambda a + \mu b + \nu c \in \mathbf{C} \mid \lambda + \mu + \nu = 1,\ \lambda, \mu, \nu > 0\}
\end{equation}
で与えられる.\\
$T$の境界$\partial T$は閉曲線として$a\to b \to c \to a$と各辺上を動くとき,
左側に$T$の内部があるように動く向きになっているとする.この向きを$\partial T$の
正の向きという.
\Lemma{
  $D$を領域,$T\subset D$を閉三角形とし,$\partial T$を上のようにする.このとき任意の
  $D$上で正則な$f$に対して,
  \begin{equation}
    \int_{\partial T}f(z)dz = 0
  \end{equation}
}{
  後で書く.
}

\Theorem{
  領域$D$内で,始点$w_{0}\in D$,終点$w_{1} \in D$を共有する二つの曲線$\gamma,\gamma'$
  が,ホモトープならば
  \begin{equation}
    \int_{\gamma}f(z)dz = \int_{\gamma'}f(z)dz
  \end{equation}
}{}