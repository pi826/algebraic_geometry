
\Section{Holomorphic Function}
\noindent
複素数$z = x + iy\in \mathbf{C}$に対して$|z|$を
\begin{equation}
  |z| \defi \sqrt{x^{2} + y^{2}}
\end{equation}
と定義する.定義とコーシー・シュワルツの不等式から$z,w\in \mathbf{C}$に対して
\begin{align}
  &|z| = 0 \Leftrightarrow z = 0\\
  &|zw| = |z||w|\\
  &|z + w| \leq |z| + |w|\qquad (三角不等式)
\end{align}
が成り立つ.三角不等式より
\begin{equation}
  ||z|-|w|| \leq |z - w|
\end{equation}
が成り立つ.実際
\begin{align*}
  |z| = |z - w + w| \leq |z - w| + |w| 
\end{align*}
より$|z| - |w| \leq |z - w|$で,対称性から$-(|z| - |w|) \leq |z-w|$より式(B.5)を
得る.\\
集合$\Omega \subset \mathbf{C}$上で定義された関数
$f:\Omega \subset \mathbf{C} \to \mathbf{C}$が$z_{0}\in \Omega$で
\index{れんぞく@連続}\index{continuous}\textbf{連続(continuous)}とは,
任意の$\varepsilon \in \mathbf{R}_{>0}$に対して,ある$\delta \in \mathbf{R}_{>0}$
が存在して
\begin{equation}
  |z - z_{0}| < \delta \Rightarrow |f(z) - f(z_{0})| < \varepsilon
\end{equation}
を満たすことである.
\footnote{ただし,$\delta$を十分小さく取る.具体的には$z\in \Omega$となる程度}
これは,$\Omega$内の点列$(z_{n})_{n}$で$z_{n} \to z_{0}\ (n \to \infty)$
なる任意の点列に対して,
\begin{equation}
  \lim_{n \to \infty}f(z_{n}) = f(z_{0})
\end{equation}
が成り立つことと同値である.
$f$が$\Omega$上で連続とは$\Omega$の任意の点で連続なときを言う.\\
$f$が連続ならば実数値関数$|f|$も連続であることがわかる.実際(B.5)より
\begin{equation}
  ||f(z)| - |f(z_{0})|| \leq |f(z) - f(z_{0})|
\end{equation}
なのでよい.\\
$f$が点$z_{0}\in \Omega$で最大値をとるとは,
\begin{equation*}
  |f(z)| \leq |f(z_{0})| \qquad (z\in \Omega)
\end{equation*}
を満たすことである.同様に最小値も定義される.

\Theorem{
  コンパクト集合$\Omega$上の連続関数は有界で,$\Omega$において最大値と最小値をとる.
}{}

証明は省く.

\Definition{
  $\Omega\subset \mathbf{C}$を開集合とし,$f:\Omega \to \mathbf{C}$が$z_{0}\in \Omega$
  で\index{せいそく@正則}\index{regular}\index{non-singular}
  \textbf{正則(regular/non-singular)}であるとは,
  \begin{equation}
    \frac{f(z+h) - f(z)}{h}
  \end{equation}
  が$h\to 0$のときに極限を持つことである.ただし,(B.9)が定義されるために,
  $h\in \mathbf{C}$をゼロでない複素数で$|h|$を十分小さく$z+h\in \Omega$
  となるように取る.この極限が存在する場合$f'(z_{0})$と書いて
  $f$の$z_{0}$における微分という.ただし,この極限において$h$はすべての方向から
  $0$に近づくような複素数であることに注意.\\
  $f$が$\Omega$で正則とは,$f$が$\Omega$の任意の点で正則であるときを言う.\\
  $\Omega$が開でないときは,$\Omega$を含むある開集合上で正則で$f$が正則である
  ということにする.$f$が$\mathbf{C}$で正則のとき$f$を
  \index{せいかんすう@整関数}\index{entire function}
  \textbf{整関数}という.
}

\Proposition{
  $f,g$を$\Omega$上の正則関数とする.このとき以下が成り立つ.
  \begin{itemize}
    \item $f+g$は$\Omega$上で正則で,$(f + g)' = f' + g'$である.
    \item $fg$は$\Omega$上で正則で,$(fg)' = f'g + fg'$である.
    \item $g(z_{0}) \neq 0$ならば$f/g$は$z_{0}$で正則で,
    \begin{equation*}
      \left( \frac{f}{g} \right)' = \frac{f'g - fg'}{g^{2}}
    \end{equation*}
    \item もし$f:\Omega \to U,g:U\to \mathbf{C}$が正則なら,
    \begin{equation*}
      (g\circ f)'(z) = g'(f(z))f'(z)\qquad (z \in \Omega)
    \end{equation*}
    が成り立つ.
  \end{itemize}
}{}

\Example{
  $f(z) = z$は正則で,$f'(z) = 1$である.よってProp:\ref{Prop:B.1.2}より,
  多項式
  \begin{equation*}
    p(z) = \sum_{i=0}^{n}a_{i}x^{i}
  \end{equation*}
  は$\mathbf{C}$で正則で,
  \begin{equation*}
    p'(z) = \sum_{i=1}^{n}i a_{i}x^{i-1}
  \end{equation*}
  である.
}{}

\Example{
  $f(z) = \bar{z}$は正則ではない.
}{}

\noindent
(B.9)の$h$が実数の場合の極限を考えよう.特に$h = h_{1} + ih_{2}$として$h_{2} = 0$
の場合を考えよう.このとき$f(z) = f(x,y)$と書くことにすると,
\begin{equation}
  f'(z) = \lim_{h_{1} \to 0}\frac{f(x + h_{1},y) - f(x,y)}{h_{1}} = \frac{\partial f}{\partial x}(z)
\end{equation}
次に$h_{1} = 0$として考えると,
\begin{equation}
  f'(z) = \lim_{h_{2} \to 0}\frac{f(x,y + h_{2}) - f(x,y)}{ih_{2}} = \frac{1}{i}\frac{\partial f}{\partial y}(z)
\end{equation}
よって,$f$が正則なら
\begin{equation}
  \frac{\partial f}{\partial x} = \frac{1}{i}\frac{\partial f}{\partial y}
\end{equation}
が成り立つ.$f = u + iv$とかけば,
\begin{equation}
  \frac{\partial u}{\partial x} = \frac{\partial v}{\partial y},\quad \frac{\partial u}{\partial y} = - \frac{\partial v}{\partial x}
\end{equation}
が成り立つことがわかる.
これらの関係式を\index{こーしーりーまんのほうていしき@コーシー・リーマンの方程式}
\index{Cauchy-Riemann equations}
\textbf{コーシー・リーマンの方程式(Cauchy-Riemann equations)}という.
ここで,ウェルティンガーの微分を定義する.
\begin{equation}
  \frac{\partial}{\partial z}\defi \frac{1}{2}\left( \frac{\partial}{\partial x} + \frac{1}{i}\frac{\partial}{\partial y} \right),\quad \frac{\partial}{\partial \bar{z}} \defi \frac{1}{2}\left( \frac{\partial}{\partial x} - \frac{1}{i}\frac{\partial}{\partial y} \right)
\end{equation}

\Proposition{
  $f$が$z_{0}$で正則なら
  \begin{equation}
    \frac{\partial f}{\partial \bar{z}}(z_{0}) = 0 
  \end{equation}
}{
  $f = u + iv$と置くと,
  \begin{align*}
    \frac{\partial f}{\partial \bar{z}} 
    &= \frac{1}{2}\left( \frac{\partial f}{\partial x} - \frac{1}{i}\frac{\partial f}{\partial y} \right)\\
    &= \frac{1}{2}\left( \frac{\partial u}{\partial x} + i\frac{\partial v}{\partial x} - \frac{1}{i}\left( \frac{\partial u}{\partial y} + i\frac{\partial v}{\partial y} \right) \right)\\
    &= \frac{1}{2} \left( \left( \frac{\partial u}{\partial x} - \frac{\partial v}{\partial y} \right) +i\left( \frac{\partial u}{\partial y} + \frac{\partial v}{\partial xL} \right)  \right)
  \end{align*}
  よってコーシー・リーマンの方程式と同値である.
}


