
\Section{Cauchy's Integral Formula}
\Theorem{(Cauchy's Integral Formula1)\\
  $f$を閉円板$\overline{D(a;r)}$の近傍上の正則関数とし,円$C(a;r)$に
  反時計回りの向き(正の向き)を与える.このとき,次の等式が成り立つ.
  \begin{equation}
    f(z) = \frac{1}{2\pi i}\int_{C(a;r)}\frac{f(\zeta)}{\zeta - z}d\zeta\qquad (z\in D(a;r))
  \end{equation}
}{
  平行移動で$a = 0$とする.ある開円板$D(r')\ (r' > r)$上で
  $f$は正則である.$z\in D(r)$に対し$\varepsilon > 0$を$\overline{D(z;\varepsilon)}\subset D(r)$となるようにとる.
  今ここで,$C(r) = \partial D(r),C(z;\varepsilon) = \partial D(z;\varepsilon)$
  は反時計回りの向き(正の向き)をもつ円とする.$z_{1}\in \partial D(r)$
}
\begin{tikzpicture}
  \draw[dashed] (-1.414,-1.414) -- (1.414,1.414);
  \coordinate (z1) at (1.414,1.414);
  \coordinate (z2) at (0.407,0.407);
  \coordinate (center1) at (0,0);
  \coordinate (center2) at (-0.3,-0.3);

  \fill (z1) circle (0.6mm) node[right] at (1.514,1.514) {$z_{1}$};
  \fill (z2) circle (0.6mm) node[right] {$z_{2}$};
  \draw[thick,
        postaction={decorate},
        decoration={markings,
                    mark=at position 0.4 with {\arrow{>}},
                    mark=at position 0.9 with {\arrow{>}}
                    }
        ](center1) circle[radius = 2] node[right] at (2,0) {$\scriptstyle \partial D(r)$};
  \draw[thick,
        postaction={decorate},
        decoration={markings,
                    mark=at position 0.5 with {\arrow{<}},
                    mark=at position 1 with {\arrow{<}}
                    }
        ](center2) circle[radius = 1] node[right] at (0.5,-0.7) { $\scriptstyle -\partial D(z;\varepsilon)$};
  \fill (center1) circle (0.6mm) node[below right] {$0$};
  \fill (center2) circle (0.6mm) node[left] {$z$};
  \draw (-1.007,-1.007) to [bend right = 60] node [fill = white, midway] {$\varepsilon$} (center2);
  \draw[thick] (z1) to (z2);
\end{tikzpicture}

