
\Section{Line Integral}
\Definition{
  \index{ぱらめとりっくきょくせん@パラメトリック曲線}\index{parametric curve}
  \textbf{パラメトリック曲線(parametric curve)}とは,写像
  $c:[a,b]\subset \mathbf{R} \to \mathbf{C}$のことである.\\
  パラメトリック曲線が\index{なめらか@滑らか}\index{smooth}
  \textbf{滑らか(smooth)}とは,$c'(t)$があって,$[a,b]$上で
  連続であり,そして$t\in [a,b]$に対して$c'(t) \neq 0$を満たすことである.
  ただし,点$t=a,t=b$に対しては$c'(a),c'(b)$を片側極限
  \begin{equation*}
    c'(a) = \lim_{\substack{h\to 0 \\ h > 0}}\frac{c(a + h) - c(a)}{h},\quad 
    c'(a) = \lim_{\substack{h\to 0 \\ h < 0}}\frac{c(b + h) - c(b)}{h}
  \end{equation*}
  で定義する.\\
  パラメトリック曲線が\index{くぶんてきになめらか@区分的に滑らか}\index{piecewise smooth}
  \textbf{区分的に滑らか(piecewise smooth)}とは,$c$が$[a,b]$上で連続で,ある
  区間の分割
  \begin{equation*}
    a = a_{0} < a_{1} < \dots < a_{n} = b
  \end{equation*}
  があって,$c(t)$が各区間$[a_{i},a_{i+1}]$において滑らかになっていることである.
  点$a_{i}$における右側微分と左側微分とは一致しないこともありうる.
}