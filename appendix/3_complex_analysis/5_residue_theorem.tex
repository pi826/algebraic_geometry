
\Section{Residue Theorem}

\Definition{
  領域$D$で正則な関数$f$が点$c\in D$で$f(c) = 0$ならば,$c$を$f$の\textbf{零点(zoro of $f$)}
  という.一致の定理によれば,$f$が恒等的に$0$でなければ,$f$の零点集合は内点を
  持たない.$c$を$f$の零点として$z = c$における$f$の冪級数展開を
  \begin{equation}
    f(z) = \sum_{n = 0}^{\infty}a_{n}(z - c)^{n}
  \end{equation}
  とする.$f(c) = 0$だから$a_{0} = 0$である.$f$は恒等的に$0$でないとするならば,
  $a_{n}$の中に$0$でないものがあるからそのような$a_{n}$の中で$n$が最小のものを
  $a_{k}$とする.このとき$f$は$c$において$k$位の零(zero of order $k$)を
  持つという.$k$を零点$c$の位数(order)または重複度(multiplicity)ということもある.
  定理〇〇によって,これは
  \begin{equation}
    f(c) = f'(c) = \dots = f^{(k-1)}(c) = 0\qquad (f^{(k)}(c)\neq 0)
  \end{equation}
  ということと同値である.
}

\Theorem{
  $f$は$D$上正則な関数で$z = c$において$k$位の零を持つとする.
  このとき$f(z) = (z - c)^{k}g(z),\ g(c) \neq 0$である$D$上の正則な関数$g$が存在する.
}{
  \begin{equation}
    g(z) = \frac{f(z)}{(z - c)^{k}}
  \end{equation}
  と定義すれば,$g$は$D\mysetminus \{c\}$で正則である.$f$の冪級数展開(B.39)を用いると,
  \begin{equation}
    \phi(z) := \frac{f(z)}{(z - c)^{k}} = \sum_{n = k}^{\infty}a_{n}(z - c)^{n - k}
  \end{equation}
  $\phi$はもとの冪級数と同じ収束半径を持つ.$\phi$の作り方から$\phi$は収束する領域内で
  $z \neq c$なる$z$に対して$g(z) = \phi(z)$が成立する.
  したがって,$g(c) = \phi(c)(=a_{k})$と定義すれば,$g$は$D$全体で正則で
  $f(z) = (z - c)^{k}g(z)$かつ$g(c)\neq 0$を満たす.
}

\Corollary{
  $f$が恒等的に$0$でないならば$f$の零点集合は離散的
  \footnote{離散集合とは,孤立点からなる集合のことを言う.}
  である.
}{
  $c$を$f$の零点とすれば,$f(z) = (z - c)^{k}g(z),\ g(c) \neq 0$なる$g$がある.
  $g(c) \neq 0$だから$c$の十分小さい近傍$U(c)$をとれば,$U(c)$の任意の点$z$に対して
  $g(z) \neq 0$である.したがって,$f$は$U(c)$内に$c$以外の零点を持たない.
  % 一致の定理から言える.
}

\Definition{
  領域$D$と$D$の離散部分集合$\{c_{\alpha}\}_{\alpha\in A}$を考える.
  $f$は$D\mysetminus \{c_{\alpha}\}_{\alpha\in A}$で正則な関数とする.
  このとき$f$が$D$上の\textbf{有理型関数(meromorphic function)}であるとは,
  各$\alpha$に対して$c_{\alpha}$の近傍$U_{\alpha}$と$U_{\alpha}$で定義された
  正則関数$g_{\alpha},h_{\alpha}$が存在して,$U_{\alpha}\mysetminus c_{\alpha}$
  上で,
  \begin{equation}
    f(z) = \frac{g_{\alpha}(z)}{h_{\alpha}(z)}
  \end{equation}
  が成り立つことである.$z = c_{\alpha}$における$g_{\alpha},h_{\alpha}$の零点の位数を
  $k,l$とする.すると定理〇〇より
  \begin{equation}
    f(z) = \frac{g_{\alpha}(z)}{h_{\alpha}(z)} = \frac{(z - c_{\alpha})^{k}g'_{\alpha}(z)}{(z - c_{\alpha})^{l}h'_{\alpha}(z)}
  \end{equation}
  で$g'_{\alpha},h'_{\alpha}$は$z = c_{\alpha}$で$0$ではないようにとれる.
  よって,$k\geq l$ならば$f$は$z = c_{\alpha}$で正則になるように拡張できる.
  逆に$k < l$ならば$f$を$z = c_{\alpha}$まで正則にはできない.この場合$f$は
  $z = c_{\alpha}$で\textbf{極(pole)}を持つといって,$l - k$を極の位数(order)という.
}
$f$は$D$上の有理型関数で$z = c$で極を持つとする.$c$を中心とした十分小さい円周$C(c;\varepsilon)$をとって,積分
\begin{equation}
  \frac{1}{2\pi i}\int_{C(c;\varepsilon)}f(z)dz
\end{equation}
ただし,円周は正の向きに向き付けられているとする.Cauchyの積分定理によって,
正数$\varepsilon$が十分小さい時積分の値は$\varepsilon$のとり方によらない.
積分の値を$f$の$z = c$における\textbf{留数(residue)}といって,$\text{Res}(f,c)$とかく.\\
極の位数を$l$とすれば,$(z - c)^{l}f(z)$は点$c$まで正則に拡張できる.したがって,$f$は原点周りで,
\begin{equation}
  f(z) = \frac{a_{-l}}{(z - c)^{l}} + \frac{a_{-(l - 1)}}{(z - c)^{l - 1}} + \dots + \frac{a_{-1}}{z - c} + a_{0} + a_{1}z + a_{2}z^{2} + \dots
\end{equation}
と展開される.$0$次以上を$\varphi(z)$とおけば,
\begin{equation}
  f(z) = \sum_{n = 1}^{l}\frac{a_{-n}}{(z - c)^{n}} + \varphi(z)
\end{equation}
よって,両辺積分すれば,
\begin{equation}
  \int_{C(c;\varepsilon)}f(z)dz = \sum_{n = 1}^{l}a_{-n}\int_{C(c;\varepsilon)}\frac{1}{(z - c)^{n}}dz + \int_{C(c;\varepsilon)}\varphi(z)dz
\end{equation}
しかし$\varphi$は正則なので右辺の第二項は$0$で
\begin{equation}
  \int_{C(c;\varepsilon)}f(z)dz = \sum_{n = 1}^{l}a_{-n}\int_{C(c;\varepsilon)}\frac{1}{(z - c)^{n}}dz
\end{equation}
を得る.$z-c = \varepsilon e^{i\theta},\ (0\leq \theta \leq 2\pi)$とすれば,
\begin{align}
  \int_{C(c;\varepsilon)}\frac{1}{(z - c)^{n}}dz 
  &= \varepsilon^{1-n}i\int_{0}^{2\pi}e^{(1 - n)i\theta}d\theta\\
  &= \left\{
  \begin{alignedat}{2}
    &2\pi i \qquad &(n = 1)\\
    &0\qquad &(n \neq 1)
  \end{alignedat}
  \right.
\end{align}
である.ゆえに(B.49)の右辺で,$n = 1$の項のみが残り,
\begin{equation}
  \text{Res}(f,z) = a_{-1}
\end{equation}
となる.
\Theorem{(Residue Theorem)\\
  $D$を有界領域で$\partial D$が区分的に滑らかな単純閉曲線だとする.また$f$は$\overline{D}$
  を含む領域で定義された有利型関数で,境界$\partial D$上には極を持たないとする.
  このとき,
  \begin{equation}
    \frac{1}{2\pi i}\int_{\partial D}f(z)dz = \sum_{c}\mathrm{Res}(f,c)
  \end{equation}
  が成り立つ.ただし右辺の和は$D$内の$f$の極をすべて動く.
}{
  $\overline{D}$はコンパクトなので,$D$内の極は有限個しかない.極$c$の
  まわりに十分小さい円周を考えてCauchyの積分定理を適用すれば求める等式を得る.
}


